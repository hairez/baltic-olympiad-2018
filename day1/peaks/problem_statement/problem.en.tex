\ifx\problemname\undefined
\providecommand\sampleinputname{}
\providecommand\sampleoutputname{}
\documentclass[english]{templates/boi}
\problemlanguage{.en}
\fi
\newcommand{\boi}{Baltic Olympiad in Informatics}
\newcommand{\practicesession}{Practice Session}
\newcommand{\contestdates}{April 27 - May 1, 2018}
\newcommand{\dayone}{Day 1}
\newcommand{\daytwo}{Day 2}
\newcommand{\licensingtext}{This problem is licensed under CC BY-SA 4.0.}
\newcommand{\problem}{Problem}
\newcommand{\inputsection}{Input}
\newcommand{\outputsection}{Output}
\newcommand{\interactivity}{Interactivity}
\newcommand{\grading}{Grading}
\newcommand{\scoring}{Scoring}
\newcommand{\constraints}{Constraints}
\renewcommand{\sampleinputname}{Sample Input}
\renewcommand{\sampleoutputname}{Sample Output}
\newcommand{\sampleexplanation}[1]{Explanation of Sample #1}
\newcommand{\sampleexplanations}{Explanation of Samples}
\newcommand{\timelimit}{Time Limit}
\newcommand{\memorylimit}{Memory Limit}
\newcommand{\seconds}{s}
\newcommand{\megabytes}{MB}
\newcommand{\group}{Group}
\newcommand{\points}{Points}
\newcommand{\limitsname}{Limits}
\newcommand{\additionalconstraints}{Additional Constraints}
\newcommand{\testgroups}{
Your solution will be tested on a set of test groups, each worth a number of points.
Each test group contains a set of test cases.
To get the points for a test group you need to solve all test cases in the test group.
Your final score will be the maximum score of a single submission.
}

\def\version{jury-draft}
\problemname{Peak Finding}
% TODO: reword this to have a more interesting story, and also to be more coherent (I'm much too tired right now...)

Given is a three-dimensional array $a$ of size $N \times M \times K$, containing integers within the range $1 \dots 10^9$.
You want to find a local maximum of the array: a value such that $a[i][j][k] \ge \max(a[i+1][j][k], a[i-1][j][k], a[i][j+1][k], a[i][j-1][k], a[i][j][k-1], a[i][j][k+1])$,
where a value is treated as $-\infty$ when it is out-of-bounds.

However, the array might be very large!
Instead of reading the entire input, you are allowed to interact with the grader, and ask for the values at given points.
When you have found a local maximum, prove that by showing it to the grader.

\section*{\interactivity}
The first line of the input will contain three positive integers: $N$, $M$ and $K$.
After that, you can print a limited number of lines of either of the two following forms of queries:
\begin{itemize}
  \item \texttt{? x y z}: This asks for the value of the array at point $(x, y, z)$.
    The grader will then reply with a single line containing the integer \texttt{a[x][y][z]}.
  \item \texttt{! x y z}: This claims that the point $(x, y, z)$ is a local maximum.
    This type of query must occur exactly once, as the very last line.
    The grader will not give a response to it.
\end{itemize}

All provided values of $x, y, z$ must obey $1 \le x \le N$, $1 \le y \le M$, $1 \le z \le K$.
If they do not, or if the input format of some line is invalid, your program will get the verdict \texttt{Wrong Answer}.
% TODO: test that it doesn't result in TLE

After each printed line, you must remember to \textbf{flush standard output} before reading the grader's reply. This is done as follows in the various allowed languages:
% TODO: test performance, give tips
\begin{itemize}
  \item C++: \texttt{std::cout << std::flush;}. \texttt{std::cout << std::endl;} also flushes, in addition to printing a new line.
  \item Java: TODO % passing true to PrintWriter, or calling flush() on some interface
  \item Pascal: \texttt{Flush(Output);}
  \item Python: TODO % print() flushes
\end{itemize}
If you fail to do this, it may result in getting a \texttt{Time Limit Exceeded} status.

\section*{\constraints}
\testgroups

% TODO: Make this into just 4 testgroups, with partial scores?
\noindent
\begin{tabular}{| l | l | l | l |}
\hline
\group & \points & \limitsname & \additionalconstraints \\ \hline
1      & 10     & $M = K = 1$, $N = 10^6$ & You can make at most $50\,000$ queries. \\ \hline
2      & 10     & $M = K = 1$, $N = 10^6$ & You can make at most $40$ queries. \\ \hline
3      & 10     & $K = 1$, $N = M = 10^3$ & You can make at most $10^5$ queries. \\ \hline
4      & 10     & $K = 1$, $N = M = 10^3$ & You can make at most $10^4$ queries. \\ \hline
5      & 10     & $K = 1$, $N = M = 10^3$ & You can make at most $4 \cdot 10^3$ queries. \\ \hline
6      & 20     & $K = N = M = 10^2$ & You can make at most $10^5$ queries. \\ \hline
7      & 30     & $K = N = M = 10^3$ & You can make at most $3 \cdot 10^5$ queries. \\ \hline
\end{tabular}
