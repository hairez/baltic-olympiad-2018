\ifx\boi\undefined\ifx\problemname\undefined
\providecommand\sampleinputname{}
\providecommand\sampleoutputname{}
\documentclass[english]{templates/boi}
\problemlanguage{.en}
\fi
\newcommand{\boi}{Baltic Olympiad in Informatics}
\newcommand{\practicesession}{Practice Session}
\newcommand{\contestdates}{April 27 - May 1, 2018}
\newcommand{\dayone}{Day 1}
\newcommand{\daytwo}{Day 2}
\newcommand{\licensingtext}{This problem is licensed under CC BY-SA 4.0.}
\newcommand{\problem}{Problem}
\newcommand{\inputsection}{Input}
\newcommand{\outputsection}{Output}
\newcommand{\interactivity}{Interactivity}
\newcommand{\grading}{Grading}
\newcommand{\scoring}{Scoring}
\newcommand{\constraints}{Constraints}
\renewcommand{\sampleinputname}{Sample Input}
\renewcommand{\sampleoutputname}{Sample Output}
\newcommand{\sampleexplanation}[1]{Explanation of Sample #1}
\newcommand{\sampleexplanations}{Explanation of Samples}
\newcommand{\timelimit}{Time Limit}
\newcommand{\memorylimit}{Memory Limit}
\newcommand{\seconds}{s}
\newcommand{\megabytes}{MB}
\newcommand{\group}{Group}
\newcommand{\points}{Points}
\newcommand{\limitsname}{Limits}
\newcommand{\additionalconstraints}{Additional Constraints}
\newcommand{\testgroups}{
Your solution will be tested on a set of test groups, each worth a number of points.
Each test group contains a set of test cases.
To get the points for a test group you need to solve all test cases in the test group.
Your final score will be the maximum score of a single submission.
}
\fi
\def\version{jury-draft}
\problemname{Peak Finding}
% TODO: reword this to have a more interesting story, also proofread

Given is a three-dimensional array $a$ of size $N \times M \times K$, containing integers within the range $1 \dots 10^9$.
You want to find a local maximum of the array: a value such that $a[i][j][k] \ge \max(a[i+1][j][k], a[i-1][j][k], a[i][j+1][k], a[i][j-1][k], a[i][j][k-1], a[i][j][k+1])$,
where a value is treated as $-\infty$ when it is out-of-bounds.

However, the array might be very large!
Instead of reading the entire input, you are allowed to interact with the grader, and ask for the values at given points.
When you have found a local maximum, prove that by showing it to the grader.

\section*{\interactivity}
The first line of the input will contain three positive integers: $N$, $M$, $K$ and $Q$.
After that, you can print at most $Q$ lines of the form \texttt{? x y z}.
This asks for the value of the array at point $(x, y, z)$.
For each such line, the grader will in response print a single line with the integer \texttt{a[x][y][z]}, which can be read by your program.

After all these lines, you program must print exactly one line of the form \texttt{! x y z}.
This claims that the point $(x, y, z)$ is a local maximum.
The grader will provide no response to this output.

All values of $x, y, z$ must obey $1 \le x \le N$, $1 \le y \le M$, $1 \le z \le K$.
If they do not, or some line has an invalid format, or you ask for more than $Q$ values,
the grader will respond with \texttt{-1} and exit.
If this happens your program should also exit. If it continues, it may incorrectly get
a verdict of \emph{Runtime Error} or \emph{Time Limit Exceeded}.

You \emph{must} make sure to flush standard output before reading the grader's response.
Additionally, this problem requires fast IO, which is non-trivial in Java and Python.
Thus, we provide helper code for interacting with the grader, which you may copy into your program.
A link to this code for all supported languages (C++, Pascal, Java, Python) can
be found in the sidebar of the kattis statement page.
We \emph{strongly} recommend using this code.

\section*{\constraints}
\testgroups

\noindent
\begin{tabular}{| l | l | l |}
\hline
\group & \points & \limitsname \\ \hline
1      & 20     & $M = K = 1$, $N = 10^6$, $Q = 10\,000$      \\ \hline
2      & 10     & $K = 1$, $N = M = 10^3$, $Q = 10^5$         \\ \hline
3      & 20     & $K = 1$, $N = M = 10^3$, $Q = 4 \cdot 10^3$ \\ \hline
4      & 20     & $K = N = M = 10^2$,      $Q = 10^5$         \\ \hline
5      & 30     & $K = N = M = 800$,       $Q = 3 \cdot 10^5$ \\ \hline
\end{tabular}
