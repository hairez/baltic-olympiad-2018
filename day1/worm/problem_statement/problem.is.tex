\ifx\boi\undefined\ifx\problemname\undefined
\providecommand\sampleinputname{}
\providecommand\sampleoutputname{}
\documentclass[icelandic]{templates/boi}
\problemlanguage{.is}
\fi
\newcommand{\boi}{Eystrasaltsólympíuleikarnir í forritnu}
\newcommand{\practicesession}{Æfingakeppni}
\newcommand{\contestdates}{27. apríl - 1. maí, 2018}
\newcommand{\dayone}{Dagur 1}
\newcommand{\daytwo}{Dagur 2}
\newcommand{\licensingtext}{This problem is licensed under CC BY-SA 4.0.}
\newcommand{\problem}{Verkefni}
\newcommand{\inputsection}{Inntak}
\newcommand{\outputsection}{Úttak}
\newcommand{\interactivity}{Gagnvirkni}
\newcommand{\grading}{Flokkun}
\newcommand{\scoring}{Stigagjöf}
\newcommand{\constraints}{Takmarkanir}
\renewcommand{\sampleinputname}{Sýnidæmis inntak}
\renewcommand{\sampleoutputname}{Sýnidæmis úttak}
\newcommand{\sampleexplanation}[1]{Útskýring á sýnidæmi #1}
\newcommand{\sampleexplanations}{Útskýringar á sýnidæmum}
\newcommand{\timelimit}{Tímamörk}
\newcommand{\memorylimit}{Minnismörk}
\newcommand{\seconds}{s}
\newcommand{\megabytes}{MB}
\newcommand{\group}{Hópur}
\newcommand{\points}{Stig}
\newcommand{\limitsname}{Takmarkanir}
\newcommand{\additionalconstraints}{Auka takmarkanir}
\newcommand{\testgroups}{
Lausnin þín verður prófuð á einhvern fjölda prufuhópa, hver hópur gefur einhvern fjölda stiga.
Hver hópur inniheldur einhvern fjölda prufutilvika.
Til að fá stig fyrir hóp þarftu að leysa öll prufutilvik innan hópsins.
Lokastigin eru fengin úr skilunum sem gáfu hæst stig.
}
\fi
\def\version{jury-1}
\problemname{Ormaáhyggjur}

Þú ert að leita að stað í jarðveginum til að setja niður orminn þinn, Maximus. Þú takmarkar leit þína
í $N \times M \times K$ sentímetra réttstrending sem þú hefur skipt upp í þrívítt fylki af 
tenings-sentímetra staðsetningum, merktar $(x,y,z)$ eftir því hvar þær eru í fylkinu 
($1 \le x \le N$, $1 \le y \le M$, $1 \le z \le K$). Hver staðsetning hefur sérstakt rakastig 
$H[x,y,z]$ sem er heiltala á bilinu $1 \dots 10^9$. Þú getur mælt rakastig hverrar staðsetningar
með sértökum skynjara.

Maximum elskar rakamikla staði, þannig þú þarft að setja hann á staðsetningu sem er í minnsta
lagi jafn rakamikil og nágrannar hennar, annars fer hann þaðan og þá verður erfitt fyrir þig
að finna hann. Í öðrum orðum þarftu að setja Maximus í staðbundið hágildi.
Í raun og veru þarftu að finna staðsetningu $(x,y,z)$ þannig að
$$
H[x,y,z] \ge \max(H[x+1,y,z], H[x-1,y,z], H[x,y+1,z], H[x,y-1,z], H[x,y,z+1], H[x,y,z-1]),
$$
þar sem að gildi er talið vera $0$ ef það er fyrir utan réttstrendinginn (því Maximus vill 
alls ekki fara fyrir utan réttstrendinginn).

Fjöldi staðsetninga getur verið frekar mikill og því viltu ekki mæla rakastig allra staðsetninga.
Þess vegna máttu tala við stiggjafarforritið og biðja um rakastigið á staðsetningum.
Þegar þú hefur fundið hæfilega staðsetningu fyrir Maximum þá gefuru stiggjafarforritinu staðsetninguna.

\section*{\interactivity}
Fyrsta línan af inntakinu inniheldur fjórar jákvæðra heiltölur: $N$, $M$, $K$ og $Q$, þar sem $N$, $M$ og $K$ eru stærðirnar á réttstrendingnum og $Q$ er mesti fjöldi mælinga semþú mátt framkvæma.

Eftir það máttu skrifa út mesta lagi $Q$ línur á forminu ``\texttt{?\ x y z}''.
Þessi skipun biður um rakastigið á staðsetningu $(x, y, z)$.
Fyrir hverja svona línu mun stiggjafarforritið svara með einni línu sem inniheldur heiltöluna $H[x,y,z]$ sem forritið þitt getur lesið inn.

Eftir allar þessar línur skal forritið þitt skrifa út nákvæmlega eina lína á forminu ``\texttt{!\ x y z}'' og hætta keyrslu.
Þetta segir að staðsetningin $(x, y, z)$ sé hæfileg staðsetning fyrir Maximus samkvæmt skilyrðinu sem var lýst að ofan.
Stigagjafarforritið svarar ekki þessarri skipun.

Öll gildi af $x, y, z$ þurfa að uppfylla $1 \le x \le N$, $1 \le y \le M$, $1 \le z \le K$.
Ef þau gera það ekki, eða ef einhver lína fylgir ekki réttu formi eða ef þú biður um meira en $Q$ rakastig
þá mun stiggjafarforritið svara með \texttt{-1} og hætta keyrslu.
Ef þetta gerist skal forritið þitt einnig hætta keyrslu. Ef forritið heldur 
áfram keyrslu gæti það fengið Runtime Error eða Time Limit Exceeded niðurstöðu.

Þú \emph{þarft} að passa að sturta úttakinu út áður en þú lest svarið frá yfirferðarforritinu,
annars mun forritið þitt fá niðurstöðuna Time Limit Exceeded. Þetta virkar í studdum málum á eftirfarandi hátt:
\begin{itemize}
  \item Java: \texttt{System.out.println()} sturtar sjálfkrafa.
  \item Python: \texttt{print()} sturtar sjálfkrafa.
  \item C++: \texttt{cout << endl;} sturtar, og fer einnig í næstu línu. Ef printf er notað, \texttt{fflush(stdout)}.
  \item Pascal: \texttt{Flush(Output)}.
\end{itemize}


Til að hjálpa til við þessi samskipti bjóðum við uppá kóða sem þú mátt afrita í forritið þitt.
Hlekkur á þennan kóða fyrir öll studd mál má finna á hliðinni á síðunni fyrir verkefnalýsinguna. Hjálparkóðinn notar einnig hratt inntak og úttak, sem getur verið gagnlegt fyrir Python og Java (skiptir bara máli í síðustu tveim prófunarhópunum).

%We especially recommend using this code if using Java or Python, whose default
%input/output (IO) routines \emph{may not be fast enough for the last two test groups}.
%The helper code uses optimized IO routines which are sufficiently fast.

Stiggjafarforritið er \emph{ekki með aðlögunarhæfni}, þ.e.a.s. hvert prófunartilvik er með fyrirfram ákveðin rakastig
og breytast þau ekki eftir því hvaða mælingar eru framkvæmdar.

\section*{\constraints}
\testgroups

\noindent
\begin{tabular}{| l | l | l |}
\hline
\group & \points & \limitsname \\ \hline
1      & 10     & $M = K = 1$, $N = 1\,000\,000$, $Q = 10\,000$  \\ \hline
2      & 22     & $M = K = 1$, $N = 1\,000\,000$, $Q = 35$       \\ \hline
3      & 12     & $K = 1$, $N = M = 200$,         $Q = 4\,000$   \\ \hline
4      & 19     & $K = 1$, $N = M = 1\,000$,      $Q = 3\,500$   \\ \hline
5      & 14     & $N = M = K = 100$,              $Q = 100\,000$ \\ \hline
6      & 23     & $N = M = K = 500$,              $Q = 150\,000$ \\ \hline
\end{tabular}

\section*{Sample dialogue}
Á Kattis er aðeins eitt sýnidæmi.
Í þessu sýnidæmi er réttstrendingurinn með stærðina $3 \times 1 \times 1$ og rakastigin á staðsetningunum eru \{10, 14, 13\}.
Að neðan má sjá sýnidæmi um samskipti lausnarforrits og stiggjafarforrits þar sem línurnar sem byrja á DÓMARI eru úttakið frá Kattis (þ.e.\ inntakið í forritið þitt), og línurnar sem byrja á ÞÚ er úttakið á þínu forriti.

Þar sem $14$ er stærri eða jafnt og nágranna gildin ($10$ og $13$), því er staðsetningin $(2,1,1)$ hæfileg staðsetning fyrir Maximus og notaðir þú þrjár mælingar sem var mesti leyfði fjöldi ($Q=3$). Þess vegna fá þessi samskipti Accepted á þessu sýnidæmi.

\begin{verbatim}
DÓMARI:   3 1 1 3
ÞÚ:       ? 3 1 1
DÓMARI:   13
ÞÚ:       ? 2 1 1
DÓMARI:   14
ÞÚ:       ? 1 1 1
DÓMARI:   10
ÞÚ:       ! 2 1 1
\end{verbatim}
