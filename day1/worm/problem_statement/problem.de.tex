\ifx\boi\undefined\ifx\problemname\undefined
\providecommand\sampleinputname{}
\providecommand\sampleoutputname{}
\documentclass[english]{templates/boi}
\problemlanguage{.en}
\fi
\newcommand{\boi}{Baltic Olympiad in Informatics}
\newcommand{\practicesession}{Practice Session}
\newcommand{\contestdates}{April 27 - May 1, 2018}
\newcommand{\dayone}{Day 1}
\newcommand{\daytwo}{Day 2}
\newcommand{\licensingtext}{This problem is licensed under CC BY-SA 4.0.}
\newcommand{\problem}{Problem}
\newcommand{\inputsection}{Input}
\newcommand{\outputsection}{Output}
\newcommand{\interactivity}{Interactivity}
\newcommand{\grading}{Grading}
\newcommand{\scoring}{Scoring}
\newcommand{\constraints}{Constraints}
\renewcommand{\sampleinputname}{Sample Input}
\renewcommand{\sampleoutputname}{Sample Output}
\newcommand{\sampleexplanation}[1]{Explanation of Sample #1}
\newcommand{\sampleexplanations}{Explanation of Samples}
\newcommand{\timelimit}{Time Limit}
\newcommand{\memorylimit}{Memory Limit}
\newcommand{\seconds}{s}
\newcommand{\megabytes}{MB}
\newcommand{\group}{Group}
\newcommand{\points}{Points}
\newcommand{\limitsname}{Limits}
\newcommand{\additionalconstraints}{Additional Constraints}
\newcommand{\testgroups}{
Your solution will be tested on a set of test groups, each worth a number of points.
Each test group contains a set of test cases.
To get the points for a test group you need to solve all test cases in the test group.
Your final score will be the maximum score of a single submission.
}
\fi
\def\version{jury-1}
\problemname{Wurmprobleme}

Du möchtest einen Platz in der Erde für deinen Wurm Maximus finden. Die Suche begrenzt du auf ein quaderförmiges Gebiet der Größe $N \times M \times K$ Zentimeter, welches du in ein dreidimensionales Raster von einem Kubikzentimeter großen Zellen eingeteilt hast, die durch ihre Position $(x,y,z)$ im Gitter beschrieben werden ($1 \le x \le N$, $1 \le y \le M$, $1 \le z \le K$).
Jede Zelle hat eine gewisse Feuchtigkeit $H[x,y,z]$, welche eine ganze Zahl im Bereich $1 \dots 10^9$ ist.
Du kannst die Feuchtigkeit einer Zelle mit einem besonderen Sensor bestimmen.

Maximus liebt feuchte Gebiete. Daher musst du ihn in eine Zelle setzen, die mindestens so feucht ist wie ihre Nachbarzellen. Ansonsten würde er weggehen und du würdest Schwierigkeiten haben, ihn wiederzufinden.
Anders ausgedrückt musst du ihn in einem lokalem Feuchtigkeitsmaximum platzieren. Genauer gesagt, finde eine Zelle $(x,y,z)$ mit
$$
H[x,y,z] \ge \max(H[x+1,y,z], H[x-1,y,z], H[x,y+1,z], H[x,y-1,z], H[x,y,z+1], H[x,y,z-1]),
$$
wobei Zellen außerhalb des Quaders den Wert $0$ haben (weil Maximus unbedingt im Quader bleiben will).

Allerdings kann die Anzahl an Zellen sehr groß werden, also möchtest du nicht die Feuchtigkeit aller Zellen messen.
Deshalb interagierst du mit dem Grader, und fragst nach der Feuchtigkeit an bestimmten Punkten.
Wenn du eine geeignete Position für Maximus gefunden hast, gib diese aus.

\section*{\interactivity}
Die erste Zeile des Inputs enthält vier positive ganze Zahlen: $N$, $M$, $K$ und $Q$, wobei $N$, $M$ und $K$ die Dimensionen des Quaders angeben und $Q$ die maximale Anzahl der Messungen ist, die du ausführen darfst.

Danach kannst du bis zu $Q$ Zeilen der Form ``\texttt{?\ x y z}'' auf der Standardausgabe ausgeben.
Dadurch fragst du den Grader nach der Feuchtigkeit am Punkt $(x, y, z)$.
Jede dieser Zeilen wird der Grader mit einer Zeile mit der ganzen Zahl $H[x,y,z]$ beantworten, die du über die Standardeingabe einlesen kannst. 

Danach muss dein Programm genau eine Zeile der Form ``\texttt{!\ x y z}'' ausgeben.
Dadurch behauptest du, dass der Punkt $(x, y, z)$ ein nach dem obigen Kriterium geeigneter Ort für Maximus ist.
Der Grader wird auf diese Ausgabe nicht antworten.

Alle Werte von $x, y, z$ müssen $1 \le x \le N$, $1 \le y \le M$, $1 \le z \le K$ erfüllen.
Falls sie dies nicht tun, oder falls eine Zeile der Ausgabe ein ungültiges Format hat, oder falls du mehr als $Q$ Anfragen stellst, wird der Grader mit \texttt{-1} antworten und sich beenden.
Falls dies passiert, sollte dein Programm sich auch beenden. Falls es weiterläuft, bekommt es eventuell fälschlicherweise die Bewertung Runtime Error oder Time Limit Exceeded.

Du \emph{musst} die Standardausgabe flushen, bevor du die Antwort des Graders einliest. Ansonsten bekommt dein Programm die Bewertung Time Limit Exceeded.
Du kannst die folgenden Befehle nutzen:
\begin{itemize}
  \item Java: \texttt{System.out.println()} flusht automatisch.
  \item Python: \texttt{print()} flusht automatisch.
  \item C++: \texttt{cout << endl;} flusht und schreibt zusätzlich einen Zeilenumbruch. Falls du printf nutzt: \texttt{fflush(stdout)}.
  \item Pascal: \texttt{Flush(Output)}.
\end{itemize}

Um bei dieser Interaktion zu helfen, stellen wir Hilfscode zur Verfügung, den du in dein Programm kopieren darfst.
Einen Link zu diesem Code für alle unterstützten Sprachen (C++, Pascal, Java, Python) kann in der Seitenleiste der Kattis-Aufgabenseite gefunden werden.
Der Hilfscode nutzt auch schnelle Eingabe/Ausgabe-Methoden, die für Python und Java hilfreich sein können (nur für die letzten beiden Testgruppen relevant).

Der Grader ist \emph{nicht adaptiv}; jeder Testfall hat eine festgelegte Verteilung von Feuchtigkeitswerten,
die nicht von den Anfragen des Programms abhängt.

\section*{\constraints}
\testgroups

\noindent
\begin{tabular}{| l | l | l |}
\hline
\group & \points & \limitsname \\ \hline
1      & 10     & $M = K = 1$, $N = 1\,000\,000$, $Q = 10\,000$  \\ \hline
2      & 22     & $M = K = 1$, $N = 1\,000\,000$, $Q = 35$       \\ \hline
3      & 12     & $K = 1$, $N = M = 200$,         $Q = 4\,000$   \\ \hline
4      & 19     & $K = 1$, $N = M = 1\,000$,      $Q = 3\,500$   \\ \hline
5      & 14     & $N = M = K = 100$,              $Q = 100\,000$ \\ \hline
6      & 23     & $N = M = K = 500$,              $Q = 150\,000$ \\ \hline
\end{tabular}

\section*{Beispielinteraktion}
In Beispiel 1 hat der Quader die Größe $3\times 1\times 1$ und die Feuchtigkeit in den drei Zellen ist \{10, 14, 13\}. Unten findest du eine Beispielinteraktion, wobei die mit JUDGE beginnenden Zeilen die Ausgabe des Graders (also die Eingabe deines Programms) und die mit YOU beginnenden Zeilen die Ausgabe deines Programms sind.

Da $14$ größer oder gleich den benachbarten Werten ($10$ und $13$) ist, ist die Position $(2,1,1)$ ein geeigneter Ort für Maximus. Es wurden 3 Anfragen gestellt, was der maximal erlaubten Anzahl entspricht ($Q=3$).
Daher wird dieser dialog auf dem Beispiel mit Accepted bewertet.

\begin{verbatim}
JUDGE:   3 1 1 3
YOU:     ? 3 1 1
JUDGE:   13
YOU:     ? 2 1 1
JUDGE:   14
YOU:     ? 1 1 1
JUDGE:   10
YOU:     ! 2 1 1
\end{verbatim}
