\ifx\boi\undefined\ifx\problemname\undefined
\providecommand\sampleinputname{}
\providecommand\sampleoutputname{}
\documentclass[swedish]{templates/boi}
\problemlanguage{.sv}
\fi
\newcommand{\practicesession}{Övningssession}
\newcommand{\dayone}{Dag 1}
\newcommand{\daytwo}{Dag 2}
\newcommand{\boi}{Baltic Olympiad in Informatics}
\newcommand{\contestdates}{27 april - 1 maj, 2018}
\newcommand{\licensingtext}{Detta problem är licensierat enligt CC BY-SA 4.0.}
\newcommand{\problem}{Problem}
\newcommand{\inputsection}{Indata}
\newcommand{\outputsection}{Utdata}
\newcommand{\interactivity}{Interaktivitet}
\newcommand{\grading}{Bedömning}
\newcommand{\scoring}{Poängsättning}
\newcommand{\constraints}{Begränsningar}
\renewcommand{\sampleinputname}{Exempel-indata}
\renewcommand{\sampleoutputname}{Exempel-utdata}
\newcommand{\sampleexplanation}[1]{Förklaring till Exempel #1}
\newcommand{\sampleexplanations}{Förklaring till exemplen}
\newcommand{\timelimit}{Tidsgräns}
\newcommand{\memorylimit}{Minnesgräns}
\newcommand{\seconds}{s}
\newcommand{\megabytes}{MB}
\newcommand{\group}{Grupp}
\newcommand{\points}{Poäng}
\newcommand{\limitsname}{Gränser}
\newcommand{\additionalconstraints}{Övriga begränsningar}
\newcommand{\testgroups}{
Your solution will be tested on a set of test groups, each worth a number of points. Each test group contains
a set of test cases. To get the points for a test group you need to solve all test cases in the test group. Your
final score will be the maximum score of a single submission.

Din lösning kommer att testas på en uppsättning testgrupper, var och en värd en viss poäng.
Varje testgrupp innehåller flera testfall. 
För att få poäng för en testgrupp måste du klara alla testfallen i gruppen.
Din slutgiltiga poäng på problemet kommer att vara den maximala poängen av en enda submission.
}

\fi
\def\version{jury-1}
\problemname{Mask-omsorg}

Du letar efter en plats i jorden att lägga din tama mask, Maximus. Du begränsar ditt letande till ett rätblock med dimensionerna $N \times M \times K$ centimeter, vilket du har delat in i ett tredimensionellt rutnät av kubikcentimeter-celler, benämnda $(x,y,z)$ enligt deras position i rutnätet ($1 \le x \le N$, $1 \le y \le M$, $1 \le z \le K$). Varje cell har en viss fuktighet $H[x,y,z]$ som är ett heltal mellan $1$ och $10^9$. Du kan mäta fuktigheten av valfri cell med en speciell sensor.

Maximus älskar fuktiga platser, så du måste placera honom i en cell som är minst lika fuktig som dess angränsande celler, annars kravlar han iväg och du får problem att hitta honom. Med andra ord ska du placera Maximus i ett lokalt maximum.
Mer preciserat: du ska finna en cell $(x,y,z)$, sådan att
$$
H[x,y,z] \ge \max(H[x+1,y,z], H[x-1,y,z], H[x,y+1,z], H[x,y-1,z], H[x,y,z+1], H[x,y,z-1]),
$$
där ett värde anses som $0$ om det är utanför rätblocket (eftersom Maximum absolut vill stanna inom detta).

Antalet celler kan dock vara väldigt stort, så du vill inte mäta fuktigheten i alla celler. Därför kan du i den här uppgiften interagera med Kattis och fråga efter fuktigheten i givna celler. När du har hittat en lämplig placering åt Maximus, ange denna plats till Kattis.

\section*{\interactivity}
Den första raden av indata innehåller fyra positiva heltal: $N$, $M$, $K$ och $Q$, där $N$, $M$ och $K$ är rätblockets dimensioner och $Q$ är det maximala antalet mätningar du får göra.

Efter detta kan du skriva högst $Q$ rader på formen \texttt{?\ x y z} till standard output.
Detta frågar efter värdet på fuktigheten i cellen  $(x, y, z)$.
För varje sådan rad, skriver Kattis som svar en enda rad med heltalet $H[x,y,z]$, vilket kan läsas från standard input av ditt program.

Efter alla dessa rader, måste ditt program skriva ut exakt en rad på formen \texttt{!\ x y z}.
Detta påstår att celllen $(x, y, z)$ är en lämplig plats för Maximus enligt kriteriet ovan. Kattis ger inget svar på den här raden.

Alla värden på $x, y, z$ måste uppfylla $1 \le x \le N$, $1 \le y \le M$, $1 \le z \le K$.
Om de inte gör det, eller om någon rad har ogiltigt format, eller om du frågar efter mer än $Q$ värden,
så svarar Kattis med \texttt{-1} och avslutas. 
Om detta händer ska ditt program också avsluta. Om det fortsätter köra, kan det få bedömningen Runtime Error eller Time Limit Exceeded.

Du \emph{måste} se till att flusha standard output innan du läser Kattis svar, annars kommer körningen bedömas som Time Limit Exceeded. Så här fungerar detta i de olika språken:
\begin{itemize}
  \item Java: \texttt{System.out.println()} flushar automatiskt.
  \item Python: \texttt{print()} flushar automatically.
  \item C++: \texttt{cout << endl;} flushar, utöver att skriva ny rad. Om du använder printf,  \texttt{fflush(stdout)}.
  \item Pascal: \texttt{Flush(Output)}.
\end{itemize}

För att hjälpa till att hantera den här interaktionen, tillhandahåller vi hjälpkod som du kan kopiera in i ditt program. Länk till den koden för alla supportade språk (C++, Pascal, Java, Python) finns i sidebaren till Kattis problemsida. Vi rekommenderar särskilt att använda denna kod om du skriver i Java eller Python, vars default input/output (IO) rutiner \emph{kan vara för långsamma för de två sista testgrupperna}.
Hjälpkoden använder optimerade IO-rutiner som är tillräckligt snabba.

Kattis kommer att vara \emph{icke-adaptiv}; d.v.s. varje testfall kommer att ha en fix uppsättning fuktighetsvärden som inte beror på vilka mätningar programmet efterfrågar.

\section*{\constraints}
\testgroups

\noindent
\begin{tabular}{| l | l | l |}
\hline
\group & \points & \limitsname \\ \hline
1      & 10     & $M = K = 1$, $N = 1\,000\,000$, $Q = 10\,000$  \\ \hline
2      & 22     & $M = K = 1$, $N = 1\,000\,000$, $Q = 35$       \\ \hline
3      & 12     & $K = 1$, $N = M = 200$,         $Q = 4\,000$   \\ \hline
4      & 19     & $K = 1$, $N = M = 1\,000$,      $Q = 3\,500$   \\ \hline
5      & 14     & $N = M = K = 100$,              $Q = 100\,000$ \\ \hline
6      & 23     & $N = M = K = 500$,              $Q = 150\,000$ \\ \hline
\end{tabular}

