\ifx\boi\undefined\ifx\problemname\undefined
\providecommand\sampleinputname{}
\providecommand\sampleoutputname{}
\documentclass[lithuanian]{templates/boi}
\usepackage[L7x]{fontenc}
\problemlanguage{.lt}
\fi
\newcommand{\boi}{Baltijos šalių informatikos olimpiada}
\newcommand{\practicesession}{Bandomasis turas}
\newcommand{\contestdates}{2018 m. balandžio 27 - gegužės 1 d.}
\newcommand{\dayone}{1 diena}
\newcommand{\daytwo}{2 diena}
\newcommand{\licensingtext}{This problem is licensed under CC BY-SA 4.0.}
\newcommand{\problem}{Uždavinys}
\newcommand{\inputsection}{Pradiniai duomenys}
\newcommand{\outputsection}{Rezultatai}
\newcommand{\interactivity}{Realizacija}
\newcommand{\grading}{Vertinimas(?)}
\newcommand{\scoring}{Taškų skyrimas(?)}
\newcommand{\constraints}{Ribojimai}
\renewcommand{\sampleinputname}{Pradiniai duomenys}
\renewcommand{\sampleoutputname}{Rezultatai}
\newcommand{\sampleexplanation}[1]{Pirmojo pavyzdžio paaiškinimas}
\newcommand{\sampleexplanations}{Pavyzdžių paaiškinimai}
\newcommand{\timelimit}{Laiko ribojimas}
\newcommand{\memorylimit}{Atminties ribojimas}
\newcommand{\seconds}{s}
\newcommand{\megabytes}{MB}
\newcommand{\group}{Grupė}
\newcommand{\points}{Taškai}
\newcommand{\limitsname}{Ribojimai}
\newcommand{\additionalconstraints}{Papildomi ribojimai}
\newcommand{\testgroups}{
Jūsų sprendimas bus testuojamas su keliomis testų grupėmis, kiekviena kurių vertinama tam tikru skaičiumi taškų.
Kiekvieną testų grupę sudarys keletas testų.
Taškai už testų grupę skiriami tik jei įveikiate visus tos grupės testus.
}
\fi
\def\version{jury-1}
\problemname{Kirmino rūpesčiai}

Jūsų augintinis -- kirminas Maximus. Dirvoje ieškote vietos, kur galėtumėte jį apgyvendinti.
Dirva yra dėžutės formos ir jos išmatavimai yra $N \times M \times K$ centimetrų.
Ji sudalinta į trimačius kubelius, kurie žymimi $(x,y,z)$ pagal jų poziciją dirvoje ($1 \le x \le N$, $1 \le y \le M$, $1 \le z \le K$). 
Yra žinomas kiekvieno kubelio drėgnumas $H[x,y,z]$. Tai sveikasis skaičius iš intervalo $1 \dots 10^9$. Kiekvieno kubelio drėgnumą galima išmatuoti specialiu sensoriumi.

Maximus mėgsta drėgmę, taigi reikia jį apgyvendinti tokiame kubelyje, kuris yra ne sausesnis už gretimus kubelius. Priešingu atveju jis pabėgs į drėgnesnį kubelį ir bus sunku jį surasti.
Kitaip sakant, Maximus reikia apgyvendinti lokaliame maksimume.
T.y. reikia rasti tokį kubelį $(x,y,z)$, kad galiotų:
$$
H[x,y,z] \ge \max(H[x+1,y,z], H[x-1,y,z], H[x,y+1,z], H[x,y-1,z], H[x,y,z+1], H[x,y,z-1])
$$

Jei kubelis yra už dirvos ribų, tuomet jo drėgnumą laikykite lygiu $0$ (nes Maximus jokiu būdu nenori išlįsti iš dėžutės formos dirvos).

Dirvoje yra labai daug kubelių, taigi jūs nenorite matuoti kiekvieno kubelio drėgnumo. Šiame uždavinyje jums leidžiama kreiptis į vertintoją (angl. \emph{grader}) ir užklausti, koks nurodytų kubelių drėgnumas.
Kai rasite tinkamą kubelį, kuriame galės apsigyventi Maximus, pateikite to kubelio koordinates vertintojui.

\section*{\interactivity}
Pirmoje eilutėje pateikti keturi teigiami sveikieji skaičiai: $N$, $M$, $K$ ir $Q$, kur $N$, $M$ ir $K$ yra dėžutės išmatavimai, o $Q$ -- didžiausias galimas drėgnumo matavimų, kuriuos galite atlikti, skaičius.

Po to į standartinį išvedimą galite parašyti daugiausia $Q$ eilučių tokiu formatu: \texttt{"?\ x y z"}.
Tai yra drėgnumo laukelyje $(x, y, z)$ užklausa.
Kiekvienai tokiai užklausai vertintojas pateiks atsakymą -- vieną sveikąjį skaičių  $H[x,y,z]$. Atsakymas bus pateiktas vienoje eilutėje ir jį galima perskaityti iš standartinio įvedimo.

Po užklausų Jūsų programa turi išvesti lygiai vieną eilutę formatu \texttt{"!\ x y z"} ir baigti darbą.
Tai reiškia, kad kubelis $(x, y, z)$ yra tinkamas apgyvendinti Maximus. 
Į šią eilutę vertintojas nepateiks jokio atsakymo.

Visos $x, y, z$ reikšmės turi priklausyti intervalams $1 \le x \le N$, $1 \le y \le M$, $1 \le z \le K$.
Jeigu pateiktos reikšmės nepriklauso šiems intervalams, arba eilutės (užklausos ar atsakymo) formatas neatitinka nurodyto, arba pateiksite daugiau nei $Q$ užklausų, vertintojas pateiks atsakymą \texttt{-1} ir baigs darbą.
Jei taip nutiks, jūsų programa taip pat turi baigti darbą. Kitu atveju, gali būti klaidingai pažymėta, kad viršytas laiko limitas arba įvyko vykdymo klaida.

Jūs \emph{privalote} ištuštinti standartinio išvedimo buferį (angl. \emph{flush standard output}) prieš skaitydami vertintojo atsakymą. Kitu atveju Jūsų programa bus pažymėta kaip viršijusi leidžiamą vykdymo laiką.
\begin{itemize}
  \item Java: \texttt{System.out.println()} ištuština automatiškai.
  \item Python: \texttt{print()} ištuština automatiškai.
  \item C++: \texttt{cout << endl;} ištuština ir papildomai išveda naujos eilutės simbolį. Jei naudojate printf: \texttt{fflush(stdout)}.
  \item Pascal: \texttt{Flush(Output)}.
\end{itemize}

Suprogramuoti interaktyvų bendravimą padės užduoties rengėjų parengtas kodas, kurį, jei norite, galite įkopijuoti į savo programą.
Nuorodą į šį kodą kiekvienai palaikomai programavimo kalbai (C++, Pascal, Java, Python) galite rasti kattis sistemos užduočių puslapio šoninėje juostoje.
Pateiktas kodas naudoja optimizuotas ir pakankamai greitas įvedimo/išvedimo paprogrames, kurios gali būti naudingos programuojant Python ir Java kalbomis, ir aktualios tik dviems paskutinėms testų grupėms.

Vertintojas \emph{nėra adaptyvus}: tai yra, kiekvienam testui kiekvieno kubelio drėgnumo vertės yra fiksuotos ir jos nepriklausys nuo jūsų programos užklausų. 
\section*{\constraints}
\testgroups

\noindent
\begin{tabular}{| l | l | l |}
\hline
\group & \points & \limitsname \\ \hline
1      & 10     & $M = K = 1$, $N = 1\,000\,000$, $Q = 10\,000$  \\ \hline
2      & 22     & $M = K = 1$, $N = 1\,000\,000$, $Q = 35$       \\ \hline
3      & 12     & $K = 1$, $N = M = 200$,         $Q = 4\,000$   \\ \hline
4      & 19     & $K = 1$, $N = M = 1\,000$,      $Q = 3\,500$   \\ \hline
5      & 14     & $N = M = K = 100$,              $Q = 100\,000$ \\ \hline
6      & 23     & $N = M = K = 500$,              $Q = 150\,000$ \\ \hline
\end{tabular}

\section*{Dialogo pavyzdys}
Kattis sistemoje yra vienas pavyzdinis testas.
Šiame pavyzdyje dėžutės išmatavimai yra $3\times 1\times 1$ ir drėgnumas kubeliuose yra \{10, 14, 13\}.
Žemiau pateiktas dialogo pavyzdys. Eilutės, prasidedančios žodžiu JUDGE, yra pateiktos Kattis sistemos (jos yra perskaitomos jūsų programos). Eilutės, prasidedančios YOU, yra išspausdinamos jūsų programos.

Kadangi $14$ yra didesnis arba lygus gretimoms reikšmėms ($10$ ir $13$), kubelis $(2,1,1)$ yra tinkama vieta apgyvendinti Maximus ir, kadangi jūs neviršijote maksimalaus užklausų skaičiaus($Q=3$), testas yra įveiktas.

\begin{verbatim}
JUDGE:   3 1 1 3
YOU:     ? 3 1 1
JUDGE:   13
YOU:     ? 2 1 1
JUDGE:   14
YOU:     ? 1 1 1
JUDGE:   10
YOU:     ! 2 1 1
\end{verbatim}
