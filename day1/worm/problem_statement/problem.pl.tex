\ifx\boi\undefined\ifx\problemname\undefined
\providecommand\sampleinputname{}
\providecommand\sampleoutputname{}
\documentclass[english]{templates/boi}
\problemlanguage{.en}
\fi
\newcommand{\boi}{Baltic Olympiad in Informatics}
\newcommand{\practicesession}{Practice Session}
\newcommand{\contestdates}{April 27 - May 1, 2018}
\newcommand{\dayone}{Day 1}
\newcommand{\daytwo}{Day 2}
\newcommand{\licensingtext}{This problem is licensed under CC BY-SA 4.0.}
\newcommand{\problem}{Problem}
\newcommand{\inputsection}{Input}
\newcommand{\outputsection}{Output}
\newcommand{\interactivity}{Interactivity}
\newcommand{\grading}{Grading}
\newcommand{\scoring}{Scoring}
\newcommand{\constraints}{Constraints}
\renewcommand{\sampleinputname}{Sample Input}
\renewcommand{\sampleoutputname}{Sample Output}
\newcommand{\sampleexplanation}[1]{Explanation of Sample #1}
\newcommand{\sampleexplanations}{Explanation of Samples}
\newcommand{\timelimit}{Time Limit}
\newcommand{\memorylimit}{Memory Limit}
\newcommand{\seconds}{s}
\newcommand{\megabytes}{MB}
\newcommand{\group}{Group}
\newcommand{\points}{Points}
\newcommand{\limitsname}{Limits}
\newcommand{\additionalconstraints}{Additional Constraints}
\newcommand{\testgroups}{
Your solution will be tested on a set of test groups, each worth a number of points.
Each test group contains a set of test cases.
To get the points for a test group you need to solve all test cases in the test group.
Your final score will be the maximum score of a single submission.
}
\fi
\def\version{jury-1}
\problemname{Problemy dżdżownicy}

Szukasz miejsca w ziemi aby umieścić tam swoją dżdżownicę, Maximusa. Poszukiwania ograniczyłeś do obszaru w kształcie pudełka (prostopadłościanu)
o wymiarach $N \times M \times K$ centymetrów. Pudełko podzieliłeś na trójwymiarową kratę jednocentymetrowych pól, oznaczonych
pozycją w tej kracie $(x,y,z)$ ($1 \le x \le N$, $1 \le y \le M$, $1 \le z \le K$). Każde pole ma określoną wilgotność $H[x,y,z]$, która jest liczbą całkowitą
z zakresu $1 \dots 10^9$. Możesz zmierzyć wilgotność w danym polu używając specjalistycznego urządzenia.

Maximus preferuje wilgotne miejsca, zatem musisz umieścić go w polu, które jest co najmniej tak wilgotne jak wszystkie sąsiadujące pola, w przeciwnym wypadku Maximus ucieknie i będziesz musiał go szukać.
Innymi słowy, musisz umieścić Maximusa w lokalnym maksimum.
Konkretniej, musisz znaleźć pole $(x,y,z)$, takie że
$$
H[x,y,z] \ge \max(H[x+1,y,z], H[x-1,y,z], H[x,y+1,z], H[x,y-1,z], H[x,y,z+1], H[x,y,z-1]),
$$
gdzie wartość pola poza pudełkiem to $0$ (ponieważ Maximus absolutnie chce zostać w tym pudełku).

Jednakże, liczba pól może być bardzo duża, zatem nie chcesz mierzyć wilgotności każdego z nich. Z tego powodu, w tym zadaniu będziesz komunikował
się z programem sprawdzającym, pytając o wilgotność pewnych punktów. Kiedy już znajdziesz odpowiednie miejsce dla Maximusa, podaj tę lokalizację
programowi sprawdzającemu.

\section*{\interactivity}
W pierwszym wierszu standardowego wejścia znajdują się cztery dodatnie liczby całkowite: $N$, $M$, $K$ i $Q$, gdzie $N$, $M$ oraz $K$ to rozmiary obszaru poszukiwań, a $Q$ jest maksymalną liczbą pomiarów, które możesz wykonać.

Następnie możesz wypisać co najwyżej $Q$ wierszy postaci \texttt{?\ x y z} na standardowe wyjście.
Taki wiersz oznacza zapytanie o wilgotność w polu $(x, y, z)$.
Dla każdego takiego wiersza program sprawdzający w odpowiedzi wypisze pojedynczy wiersz z liczbą całkowitą $H[x,y,z]$, 
którą Twój program może wczytać ze standardowego wejścia.

Po zadaniu wszystkich zapytań Twój program musi wypisać dokładnie jeden wiersz postaci \texttt{!\ x y z} i się zakończyć.
Taki wiersz oznacza, że pole $(x, y, z)$ jest odpowiednią lokalizacją dla Maximusa według powyższych kryteriów.
Program sprawdzający nie zwróci żadnej odpowiedzi dla takiego komunikatu.

Wszystkie wartości $x, y, z$ muszą spełniać $1 \le x \le N$, $1 \le y \le M$, $1 \le z \le K$.
Jeżeli tak nie będzie, albo pewien wiersz będzie miał niewłaściwy format, bądź zadasz więcej niż $Q$ zapytań,
program sprawdzający wypisze \texttt{-1} oraz zakończy interakcję.
W takim przypadku Twój program też powinien się zakończyć. W przeciwnym wypadku możesz niepoprawnie otrzymać werdykt
Runtime Error albo Time Limit Exceeded.

Pamiętaj, żeby opróżniać bufor wyjściowy \emph{po każdym wierszu}, przed wczytaniem odpowiedzi
sprawdzarki, inaczej twój program otrzyma wynik Time Limit Exceeded.
Komendy opróżniające bufor we wspieranych językach:
\begin{itemize}
  \item Java: \texttt{System.out.println()} robi to automatycznie.
  \item Python: \texttt{print()} również robi to sam.
  \item C++: \texttt{cout << endl;} opróżnia bufor, dodatkowo wypisując znak nowej linii. Jeżeli używasz printf: \texttt{fflush(stdout)}.
  \item Pascal: \texttt{Flush(Output)}.
\end{itemize}

Aby pomóc z radzeniem sobie z komunikacją, zapewniamy opcjonalny dodatkowy kod, który możesz skopiować
do swojego programu. Odnośnik do tego kodu dla wszystkich wspieranych języków (C++, Pascal, Java, Python)
możesz znaleźć w menu bocznym na stronie z problemami. 
Pomocniczy kod używa zoptymalizowanych metod wczytywania i wypisywania, co może być przydatne
dla Javy i Pythona w dwóch ostatnich przypadkach testowych.

Program sprawdzający \emph{nie będzie adaptacyjny}, tj. dla każdego testu będą ustalone wartości wilgotności,
które nie będą zależeć od pomiarów wykonanych przez Twój program.

\section*{\constraints}
\testgroups

\noindent
\begin{tabular}{| l | l | l |}
\hline
\group & \points & \limitsname \\ \hline
1      & 10     & $M = K = 1$, $N = 1\,000\,000$, $Q = 10\,000$  \\ \hline
2      & 22     & $M = K = 1$, $N = 1\,000\,000$, $Q = 35$       \\ \hline
3      & 12     & $K = 1$, $N = M = 200$,         $Q = 4\,000$   \\ \hline
4      & 19     & $K = 1$, $N = M = 1\,000$,      $Q = 3\,500$   \\ \hline
5      & 14     & $N = M = K = 100$,              $Q = 100\,000$ \\ \hline
6      & 23     & $N = M = K = 500$,              $Q = 150\,000$ \\ \hline
\end{tabular}

\section*{Sample dialogue}
W pierwszym teście przykładowym pudełko ma wymiary $3\times 1\times 1$, wilgotność kolejnych pól wynosi odpowiednio {10, 14, 13}.
Poniżej znajduje się przykład interakcji dla tego testu. Linie oznaczone YOU zostały wypisane przez Twój program,
a linie oznaczone JUDGE przez Kattis (wczytane przez Twój program).

$14$ jest większe bądź równe sąsiednim wartościom ($10$ i $13$), pozycja $(2,1,1)$ jest doskonałym miejscem dla Maximusa.
Twój program użył łącznie trzech zapytań, co było maksymalną możliwą liczbą zapytań w tym teście.
Program otrzyma wynik Accepted.

\begin{verbatim}
JUDGE:   3 1 1 3
YOU:     ? 3 1 1
JUDGE:   13
YOU:     ? 2 1 1
JUDGE:   14
YOU:     ? 1 1 1
JUDGE:   10
YOU:     ! 2 1 1
\end{verbatim}
