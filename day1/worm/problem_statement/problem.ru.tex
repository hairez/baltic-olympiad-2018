\ifx\boi\undefined\ifx\problemname\undefined
\providecommand\sampleinputname{}
\providecommand\sampleoutputname{}
\documentclass[russian]{templates/boi}
\problemlanguage{.ru}
\fi
\newcommand{\boi}{Балтийская Олимпиада по Информатике}
\newcommand{\practicesession}{Тренировочный раунд}
\newcommand{\contestdates}{27 апреля - 1 мая, 2018}
\newcommand{\dayone}{День 1}
\newcommand{\daytwo}{День 2}
\newcommand{\licensingtext}{Задача публикуется под лицензией CC BY-SA 4.0.}
\newcommand{\problem}{Задача}
\newcommand{\inputsection}{Ввод}
\newcommand{\outputsection}{Вывод}
\newcommand{\interactivity}{Интерактивность}
\newcommand{\grading}{Оценивание}
\newcommand{\scoring}{Очки}
\newcommand{\constraints}{Ограничения}
\renewcommand{\sampleinputname}{Пример ввода}
\renewcommand{\sampleoutputname}{Пример вывода}
\newcommand{\sampleexplanation}[1]{Объяснение примера #1}
\newcommand{\sampleexplanations}{Объяснение примеров}
\newcommand{\timelimit}{Ограничение по времени}
\newcommand{\memorylimit}{Ограничение на память}
\newcommand{\seconds}{сек}
\newcommand{\megabytes}{MB}
\newcommand{\group}{Группа}
\newcommand{\points}{Очки}
\newcommand{\limitsname}{Ограничения}
\newcommand{\additionalconstraints}{Дополнительные ограничения}
\newcommand{\testgroups}{
Тесты разделены на группы. Очки за группу даются только если корректно решены все тесты в группе.
}
\fi
\def\version{jury-1}
\problemname{Червячьи радости}

Вы решили вывести своего домашнего червячка Максимку на выгул и хотели бы найти для него оптимальное место в земле. В поиске идеального места вы решили ограничиться областью пространства в форме параллелограма размерами $N \times M \times K$ сантиметров, которую вы разделили на трехмерную сеть клеток, каждая размером в один кубический сантиметр. Каждая клетка имеет координату $(x,y,z)$ в соответствии с её положением в пространстве ($1 \le x \le N$, $1 \le y \le M$, $1 \le z \le K$). С помощью специального сенсора можно измерить влажность почвы $H[x,y,z]$ в любой клетке. Влажность выражается в виде целого числа $1 \dots 10^9$.

Максимка любит влажные места, и поэтому вы должны поместить его в клетку, влажность которой как минимум не меньше влажности всех соседних клеток (иначе он уползет в соседнюю, более влажную клетку, и его потом будет сложно найти). Другими словами, вам нужно поместить Максимку в клетку-локальный максимум по влажности.

Строго говоря, требуется найти клетку $(x,y,z)$, которая удовлетворяет условию:
$$
H[x,y,z] \ge \max(H[x+1,y,z], H[x-1,y,z], H[x,y+1,z], H[x,y-1,z], H[x,y,z+1], H[x,y,z-1]).
$$
Значение влажности вне пределов выбранной области пространства считаем равным $0$ (потому что Максимке очень хочется оставаться в этой области).

Количество клеток, однако, может быть довольно большим, и вам не хотелось бы измерять влажность для всех из них. В этой задаче вы можете взаимодействовать с программой-оценивателем, запрашивая влажность в выбранных клетках по одной. Как только вы нашли подходящее место для Максимки, сообщите его оценивателю и завершите работу.

\section*{\interactivity}
На первой строке ввода даны четыре положительные числа: $N$, $M$, $K$ и $Q$, где $N$, $M$ и $K$ -- размеры области, а $Q$ -- максимальное количество измерений, которое вы можете совершить.

После этого вы можете вывести не более $Q$ строк вида \texttt{?\ x y z} в стандартный вывод.
Каждая такая строка соответствует запросу влажности в клетке $(x, y, z)$, и в ответ на неё оцениватель выведет одну строку со значением $H[x,y,z]$, которую ваша программа может зачитать из стандартного ввода.

После совершения необходимого числа запросов, ваша программа должна вывести ровно одну строку вида \texttt{!\ x y z}.
Это соответствует утверждению, что клетка $(x, y, z)$ является подходящим местом для Максимки, согласно требуемым условиям. После этого оцениватель не будет давать какого-либо ввода.

Все значения $x, y, z$ должны удовлетворять $1 \le x \le N$, $1 \le y \le M$, $1 \le z \le K$.
Если они этому условию не удовлетворяют, нарушают формат, или же если вы совершаете более $Q$ запросов, оцениватель выдаст ответ \texttt{-1} и завершит работу.
После этого ваша программа тоже должна завершить работу. Если этого не сделать, программа может некорректно получить результат вида Runtime Error или Time Limit Exceeded.

Вы \emph{обязаны} всегда сбрасывать (flush) буфер стандартного вывода перед зачитыванием следующего ввода, иначе оцениватель может зависнуть в ожидании результата, и ваша программа получит оценку Time Limit Exceeded. Сбрасывание буфера реализуется в различных языках следующим образом:
\begin{itemize}
  \item Java: \texttt{System.out.println()} автоматически сбрасывает буфер.
  \item Python: \texttt{print()} автоматически сбрасывает буфер.
  \item C++: \verb#cout << endl;# выводит новую строку и сбрасывает буфер. При использовании printf, буфер сбрасывается командой \texttt{fflush(stdout)}.
  \item Pascal: \texttt{Flush(Output)}.
\end{itemize}

Чтобы помочь в реализации взаимодействия с оценивателем, мы предоставляем вспомогательные примеры кода, которые вы можете скопировать к себе в программу.
Ссылка на этот код для всех поддерживаемых языков (C++, Pascal, Java, Python) находится на боковой панели страницы задачи в системе Kattis.

Мы особенно рекомендуем использовать этот код, если вы программируете на Java или Python, где процедуры ввода-вывода по умолчанию \emph{могут быть недостаточно быстрыми для решения последних двух групп тестов}.
Вспомогательный код использует оптимизированные алгоритмы ввода-вывода, скорость которых достаточна для реализации корректного решения.

Алгоритм оценивания \emph{не является адаптивным}. Это значит, что каждому тесту соответствует фиксированный набор значений влажности, который не зависит от того, какие измерения совершает программа.

\section*{\constraints}
\testgroups

\noindent
\begin{tabular}{| l | l | l |}
\hline
\group & \points & \limitsname \\ \hline
1      & 10     & $M = K = 1$, $N = 1\,000\,000$, $Q = 10\,000$  \\ \hline
2      & 22     & $M = K = 1$, $N = 1\,000\,000$, $Q = 35$       \\ \hline
3      & 12     & $K = 1$, $N = M = 200$,         $Q = 4\,000$   \\ \hline
4      & 19     & $K = 1$, $N = M = 1\,000$,      $Q = 3\,500$   \\ \hline
5      & 14     & $N = M = K = 100$,              $Q = 100\,000$ \\ \hline
6      & 23     & $N = M = K = 500$,              $Q = 150\,000$ \\ \hline
\end{tabular}

\section*{Пример диалога}
В примере ниже область имеет размеры $3\times 1\times 1$, а влажность трех клеток равна {10, 14, 13}. 
Далее приведён пример диалога с оценивающей программой. Строки, начинающиеся со слова
JUDGE -- это вывод оценивателя Kattis (т.е. ввод для вашей программы). Строки, начинающиеся со слов
YOU -- это вывод вашей программы.

Т.к. $14$ -- это больше чем соседние значения ($10$ и $13$), то клетка $(2,1,1)$ является подходящим местом
для Максимки. Использовано было три запроса, что соответствует разрешенному максимуму ($Q=3$). 
Поэтому данный диалог принимается как корректное решение для данного примера.

\begin{verbatim}
JUDGE:   3 1 1 3
YOU:     ? 3 1 1
JUDGE:   13
YOU:     ? 2 1 1
JUDGE:   14
YOU:     ? 1 1 1
JUDGE:   10
YOU:     ! 2 1 1
\end{verbatim}
