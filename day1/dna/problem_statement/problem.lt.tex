\ifx\boi\undefined\ifx\problemname\undefined
\providecommand\sampleinputname{}
\providecommand\sampleoutputname{}
\documentclass[lithuanian]{templates/boi}
\usepackage[L7x]{fontenc}
\problemlanguage{.lt}
\fi
\newcommand{\boi}{Baltijos šalių informatikos olimpiada}
\newcommand{\practicesession}{Bandomasis turas}
\newcommand{\contestdates}{2018 m. balandžio 27 - gegužės 1 d.}
\newcommand{\dayone}{1 diena}
\newcommand{\daytwo}{2 diena}
\newcommand{\licensingtext}{This problem is licensed under CC BY-SA 4.0.}
\newcommand{\problem}{Uždavinys}
\newcommand{\inputsection}{Pradiniai duomenys}
\newcommand{\outputsection}{Rezultatai}
\newcommand{\interactivity}{Realizacija}
\newcommand{\grading}{Vertinimas(?)}
\newcommand{\scoring}{Taškų skyrimas(?)}
\newcommand{\constraints}{Ribojimai}
\renewcommand{\sampleinputname}{Pradiniai duomenys}
\renewcommand{\sampleoutputname}{Rezultatai}
\newcommand{\sampleexplanation}[1]{Pirmojo pavyzdžio paaiškinimas}
\newcommand{\sampleexplanations}{Pavyzdžių paaiškinimai}
\newcommand{\timelimit}{Laiko ribojimas}
\newcommand{\memorylimit}{Atminties ribojimas}
\newcommand{\seconds}{s}
\newcommand{\megabytes}{MB}
\newcommand{\group}{Grupė}
\newcommand{\points}{Taškai}
\newcommand{\limitsname}{Ribojimai}
\newcommand{\additionalconstraints}{Papildomi ribojimai}
\newcommand{\testgroups}{
Jūsų sprendimas bus testuojamas su keliomis testų grupėmis, kiekviena kurių vertinama tam tikru skaičiumi taškų.
Kiekvieną testų grupę sudarys keletas testų.
Taškai už testų grupę skiriami tik jei įveikiate visus tos grupės testus.
}
\fi
\def\version{jury-1}
\problemname{Marsiečių DNR}

Kaip tikriausiai jau žinote, žmonių DNR gali būti vaizduojami ilga eilute, kurioje galimos tik keturios raidės 
 ({A, C, G, T}). Kiekviena šių raidžių vaizduoja atitinkamą skirtingą nukleotidą (atitinkamai, adeniną, citoziną, guaniną ir timiną).

Tuo tarpu su marsiečiais yra kiek kitaip -- ištyrus paskutinį NASA pagautą marsietį paaiškėjo, kad marsiečių DNR susideda iš $K$ skirtingų nukleotidų. Todėl marsiečių DNR kodas gali būt aprašytas kaip simbolių eilutė, kurioje galimi $K$ skirtingų simbolių.

Viena tyrėjų grupė, besidominti marsiečių DNR naudojimu dirbtiniam intelektui, paprašė vientiso marsiečio DNR posekio. Be to, kiekvienam iš pasirinktų $R$ nukleotidų jie nurodė, kiek mažiausiai to nukleotido turi būti šiame posekyje.

Jus domina trumpiausias vientisas DNR posekis, tenkinantis šias sąlygas.

\section*{\inputsection}
Pirmoje eilutėje pateikti trys sveikieji skaičiai $N$, $K$ ir $R$, atitinkamai nurodantys marsiečio DNR ilgį, skirtingų nukleotidų skaičių ir skaičių nukleotidų, kuriems mokslininkai turi minimalaus kiekio reikalavimus. Šie skaičiai tenkina sąlygas $1 \le R \le K \le N$.

Antroje eilutėje yra $N$ tarpais atskirtų skaičių -- pilna marsiečio DNR seka. $i$-asis šių skaičių, $D_i$, nurodo, koks nukleotidas yra $i$-ojoje DNR sekos pozicijoje. Nukleotidai numeruojami nuo nulio, t.y., $0
\leq D_i < K$. Visi nukleotidai DNR sekoje pasitaikys bent po vieną kartą.

Kiekvienoje tolesnių $R$ eilučių pateikta po du skaičius $B$ ir $Q$, nurodančius nukleotidą ir minimalų jo kiekį ($0 \le B < K, 1 \le Q \le N$).
Šiose $R$ eilučių pasikartojančių nukleotidų nebus.

\section*{\outputsection}
Išveskite vieną sveikąjį skaičių -- trumpiausio vientiso DNR posekio, tenkinančio mokslininkų reikalavimus, ilgį. Jei tokio posekio nėra, išveskite ``\texttt{impossible}''.

\section*{\constraints}
\testgroups

\noindent
\begin{tabular}{| l | l | l |}
\hline
\group & \points & \limitsname \\ \hline
1     & 16     & $1 \le N \le 100, R \le 10$ \\ \hline
2     & 24     & $1 \le N \le 4\,000, R \le 10$ \\ \hline
3     & 28     & $1 \le N \le 200\,000, R \le 10$ \\ \hline
4     & 32     & $1 \le N \le 200\,000$ \\ \hline
\end{tabular}

\section*{\sampleexplanations}
Pirmajame pavyzdyje yra trys tinkami vientisi posekiai, kurių ilgis $2$. Jie visi turi po vieną nukleotidą $0$ ir $1$ ( ``\texttt{0 1}'', ``\texttt{1 0}'' ir ``\texttt{0 1}''),
bet nėra jokių tinkamų vientisų posekių, kurių ilgis $1$. Todėl trumpiausias ilgis yra $2$.

Antrajame pavyzdyje vienintelis optimalus posekis yra ``\texttt{1 3 2 0 1 2 0}''.

Trečiajame pavyzdyje neužtenka nulinio tipo nukleotidų.
