\ifx\boi\undefined\ifx\problemname\undefined
\providecommand\sampleinputname{}
\providecommand\sampleoutputname{}
\documentclass[english]{templates/boi}
\problemlanguage{.en}
\fi
\newcommand{\boi}{Baltic Olympiad in Informatics}
\newcommand{\practicesession}{Practice Session}
\newcommand{\contestdates}{April 27 - May 1, 2018}
\newcommand{\dayone}{Day 1}
\newcommand{\daytwo}{Day 2}
\newcommand{\licensingtext}{This problem is licensed under CC BY-SA 4.0.}
\newcommand{\problem}{Problem}
\newcommand{\inputsection}{Input}
\newcommand{\outputsection}{Output}
\newcommand{\interactivity}{Interactivity}
\newcommand{\grading}{Grading}
\newcommand{\scoring}{Scoring}
\newcommand{\constraints}{Constraints}
\renewcommand{\sampleinputname}{Sample Input}
\renewcommand{\sampleoutputname}{Sample Output}
\newcommand{\sampleexplanation}[1]{Explanation of Sample #1}
\newcommand{\sampleexplanations}{Explanation of Samples}
\newcommand{\timelimit}{Time Limit}
\newcommand{\memorylimit}{Memory Limit}
\newcommand{\seconds}{s}
\newcommand{\megabytes}{MB}
\newcommand{\group}{Group}
\newcommand{\points}{Points}
\newcommand{\limitsname}{Limits}
\newcommand{\additionalconstraints}{Additional Constraints}
\newcommand{\testgroups}{
Your solution will be tested on a set of test groups, each worth a number of points.
Each test group contains a set of test cases.
To get the points for a test group you need to solve all test cases in the test group.
Your final score will be the maximum score of a single submission.
}
\fi
\def\version{jury-1}
\problemname{Marsi DNA}
Nagu sa ilmselt tead, saab inimese DNA esitada pika stringina, kus esineb vaid
neli erinevat tähte ({A, C, G, T}) ja iga sümbol vastab erinevale nukleotiidile
(adeniin, tsütosiin, guaniin ja tümiin).

Marslastel käivad asjad aga veidi teisiti: uuringud näitavad, et viimasel NASA 
poolt kinni püütud marslasel on tervelt $K$ erinevat nukleotiidi! Seega saab marslase
DNA esitada kui stringi, mille märgid on tähestikust suurusega $K$.

Nüüd uurib grupp teadlasi, kuidas kasutada marslaste DNA-d tehisintellektirakendustes ja
neil on vaja marslase DNA näidist. Täpsemalt on nad öelnud $R$ nukleotiidi kohta,
mis on minimaalne arv, kui palju vastavat nukleotiidi peab näidises olema. 

Sul on vaja leida lühim võimalik nõuetele vastav DNA alamstring.

\section*{\inputsection}
Sisendi esimesel real on kolm täisarvu $N$, $K$ ja $R$, mis tähistavad vastavalt marslase DNA 
täielikku pikkust, tähestiku suurust ning nende nukleotiidide arvu, millele rakendub minimaalse arvu nõue. 
Kehtib reegel $1 \le R \le K \le N$.

Teisel real on $N$ tühikutega eraldatud täisarvu, mis kirjeldavad täielikku marslase DNA järjendit.
$i$-s arv $D_i$ tähistab, milline nukleotiid on marslase DNA-s $i$-ndal kohal. Nukleotiidide loendamine
algab nullist, s.t $0 \leq D_i < K$. Iga nukleotiidi esineb vähemalt üks kord.

Järgmistel $R$ real on igaühel kaks täisarvu $B$ ja $Q$, mis tähistavad nukleotiidi ja selle minimaalset
soovitud hulka ($0 \le B < K, 1 \le Q \le N$).
Ükski nukleotiid ei esine rohkem kui üks kord.

\section*{\outputsection}
Väljastada üks täisarv, lühima nõuetele vastava DNA alamstringi pikkus. Kui sellist alamstringi pole,
väljastada ``\texttt{impossible}''.

\section*{\constraints}
\testgroups

\noindent
\begin{tabular}{| l | l | l |}
\hline
\group & \points & \limitsname \\ \hline
1     & 16     & $1 \le N \le 100, R \le 10$ \\ \hline
2     & 24     & $1 \le N \le 4\,000, R \le 10$ \\ \hline
3     & 28     & $1 \le N \le 200\,000, R \le 10$ \\ \hline
4     & 32     & $1 \le N \le 200\,000$ \\ \hline
\end{tabular}

\section*{\sampleexplanations}
Esimeses näites leidub kolm alamstringi pikkusega $2$, mis kõik sisaldavad ühe nukleotiidi 0 ja ühe 1 
(täpsemalt ``\texttt{0 1}'', ``\texttt{1 0}'' ja ``\texttt{0 1}''),
aga ei leidu ühtki alamstringi pikkusega $1$. Lühim pikkus on seega $2$.

Teises näites on (ainus) optimaalne alamstring ``\texttt{1 3 2 0 1 2 0}''.

Kolmandas näites pole piisavalt 0-tüüpi nukleotiide.
