\ifx\boi\undefined\ifx\problemname\undefined
\providecommand\sampleinputname{}
\providecommand\sampleoutputname{}
\documentclass[russian]{templates/boi}
\problemlanguage{.ru}
\fi
\newcommand{\boi}{Балтийская Олимпиада по Информатике}
\newcommand{\practicesession}{Тренировочный раунд}
\newcommand{\contestdates}{27 апреля - 1 мая, 2018}
\newcommand{\dayone}{День 1}
\newcommand{\daytwo}{День 2}
\newcommand{\licensingtext}{Задача публикуется под лицензией CC BY-SA 4.0.}
\newcommand{\problem}{Задача}
\newcommand{\inputsection}{Ввод}
\newcommand{\outputsection}{Вывод}
\newcommand{\interactivity}{Интерактивность}
\newcommand{\grading}{Оценивание}
\newcommand{\scoring}{Очки}
\newcommand{\constraints}{Ограничения}
\renewcommand{\sampleinputname}{Пример ввода}
\renewcommand{\sampleoutputname}{Пример вывода}
\newcommand{\sampleexplanation}[1]{Объяснение примера #1}
\newcommand{\sampleexplanations}{Объяснение примеров}
\newcommand{\timelimit}{Ограничение по времени}
\newcommand{\memorylimit}{Ограничение на память}
\newcommand{\seconds}{сек}
\newcommand{\megabytes}{MB}
\newcommand{\group}{Группа}
\newcommand{\points}{Очки}
\newcommand{\limitsname}{Ограничения}
\newcommand{\additionalconstraints}{Дополнительные ограничения}
\newcommand{\testgroups}{
Тесты разделены на группы. Очки за группу даются только если корректно решены все тесты в группе.
}
\fi
\def\version{jury-1}
\problemname{Марсианская ДНК}
Как вам наверное известно, человеческую ДНК можно записать в виде длинной строки над алфавитом из четырех символов ({A, C, G, T}), 
где каждый символ обозначает определённое нуклеотидное основание (аденин, цитозин, гуанин и тимин, соответственно).

У марсиан, однако, всё не так. Недавние исследования над захваченным недавно NASA марсианином выявили, что
марсианская ДНК содержит $K$ различных нуклеотидных оснований! Поэтому её можно записать в виде строки над алфавитом из $K$ символов.

Некая группа исследователей, желающая начать использовать марсианскую ДНК для разработки искусственного интеллекта,
запросила отрезок последовательности марсианской ДНК. Для $R$ нуклеотидных оснований исследователи указали минимальное их количество,
которое должно присутствовать в отрезке.

Вам необходимо найти самый короткий возможный отрезок ДНК, удовлетворяющий требованиям исследователей.

\section*{\inputsection}
На первой строке даны три целых числа $N$, $K$, и $R$ ($1 \le R \le K \le N$), обозначающих суммарную длину марсианской ДНК, размер алфавита, и число нуклеотидных оснований,
для которых исследователи указали минимальное требуемое их количество.

На второй строке даны $N$ разделенных пробелами целых чисел -- вся марсианская последовательность ДНК. $i$-тый элемент $D_i$ этой последовательности  указывает нуклеотидное основание в соответствующей позиции марсианской ДНК.
Нуклеотидные основания нумеруются с $0$, т.е. $0 \leq D_i < K$. Каждое основание встречается в последовательности как минимум один раз.

На каждой из следующих $R$ строк даны два целых числа $B$ и $Q$ -- нуклеотидное основание и минимальное требуемое количество этого основания в искомом отрезке ($0 \le B < K, 1 \le Q \le N$).
Ни одно основание не будет упомянуто этом списке более одного раза.

\section*{\outputsection}
Вывести одно целое число -- длину кратчайшего отрезка (подстроки) ДНК, удовлетворяющего требованиям исследователей. Если такого отрезка не существует, вывести текст ``\texttt{impossible}''.

\section*{\constraints}
\testgroups

\noindent
\begin{tabular}{| l | l | l |}
\hline
\group & \points & \limitsname \\ \hline
1     & 16     & $1 \le N \le 100, R \le 10$ \\ \hline
2     & 24     & $1 \le N \le 4\,000, R \le 10$ \\ \hline
3     & 28     & $1 \le N \le 200\,000, R \le 10$ \\ \hline
4     & 32     & $1 \le N \le 200\,000$ \\ \hline
\end{tabular}

\section*{\sampleexplanations}
В первом примере существует три отрезка длиной $2$, которые содержат по одному вхождению нуклеотидных оснований 0 и 1 (а именно: ``\texttt{0 1}'', ``\texttt{1 0}'' и ``\texttt{0 1}''), а отрезков длиной $1$, удовлетворяющих этому условию, не существует. Поэтому ответ -- $2$.

Во втором примере, единственная оптимальная подстрока -- ``\texttt{1 3 2 0 1 2 0}''.

В третьем примере недостаточно нуклеотидных оснований типа 0.
