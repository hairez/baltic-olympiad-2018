\def\version{jury-draft}
\problemname{Alternating Current}
Fredrik is at home, playing with a custom-built model railway which he is very proud of.
The railway consists of $N$ segments connected in a circle, numbered $1, 2, \dots, N$ in clockwise order.
Electricity to the train is provided through $M$ arcs that pass along the
circle. Each segment has at least one arc that goes along it.

However, Fredrik is becoming bored with his train -- it just moves around in
circles, and he feels like it could use some more excitement.
Thus, he comes up with a plan: he will improve things by adding train switches
to all the segments along the railway. These switches will be hooked up to an
automatic system that toggles them at random, occasionally causing the train to
take detours or maybe even go off the rails completely!

Fredrik has already purchased all switches he need for this, but there is one
remaining problem: the switches also need electricity.
And not just any kind of electricity; they specifically need \emph{alternating current}.\footnote{This makes sense because the railway is a Swedish one -- in Sweden, all train switches (``växlar'') use alternating current (``växelström'').}

The way you get alternating current, Fredrik figures, is by having electricity
that goes in both directions. Thus he wants to select some of the electricity
arcs that span the railway and orient them counter-clockwise, such that every
segment is passed along by both a clockwise and a counter-clockwise arc.

Can you help Fredrik with this task?

\section*{\inputsection}
The first line contains two integers $N$ and $M$.

The next $M$ lines each contain two numbers $1 \le a, b \le N$, indicating that
there is an arc which covers segments $a, a+1, \dots, b$. If $b$ is smaller
than $a$, it means that the sequence wraps around, i.e. segments
$a, \dots, N, 1, \dots, b$ are covered.

\section*{\outputsection}
Print a single line with $M$ \texttt{0}s and \texttt{1}s, which denote whether
each segment is oriented clockwise (0) or counter-clockwise (1).

If it is impossible to orient the segments, output ``\texttt{IMPOSSIBLE}'' (without quotes).

\section*{\constraints}
Your solution will be tested on a set of test groups.
To get points for a group you need to solve all test cases in the group.

\noindent
\begin{tabular}{| l | l | l | l |}
\hline
\textbf{\group} & \textbf{\points} & \textbf{\limitsname} & \textbf{\additionalconstraints} \\ \hline
  1     & 20     & $2 \le N, M \le 20$ & \\ \hline
  2     & 20     & $2 \le N, M \le 100$ & \\ \hline
  3     & 20     & $2 \le N, M \le 1000$ & \\ \hline
  4     & 20     & $2 \le N, M \le 10^5$ & no segment has $b < a$ \\ \hline
  5     & 20     & $2 \le N, M \le 10^5$ & \\ \hline
\end{tabular}
