\ifx\boi\undefined\ifx\problemname\undefined
\providecommand\sampleinputname{}
\providecommand\sampleoutputname{}
\documentclass[russian]{templates/boi}
\problemlanguage{.ru}
\fi
\newcommand{\boi}{Балтийская Олимпиада по Информатике}
\newcommand{\practicesession}{Тренировочный раунд}
\newcommand{\contestdates}{27 апреля - 1 мая, 2018}
\newcommand{\dayone}{День 1}
\newcommand{\daytwo}{День 2}
\newcommand{\licensingtext}{Задача публикуется под лицензией CC BY-SA 4.0.}
\newcommand{\problem}{Задача}
\newcommand{\inputsection}{Ввод}
\newcommand{\outputsection}{Вывод}
\newcommand{\interactivity}{Интерактивность}
\newcommand{\grading}{Оценивание}
\newcommand{\scoring}{Очки}
\newcommand{\constraints}{Ограничения}
\renewcommand{\sampleinputname}{Пример ввода}
\renewcommand{\sampleoutputname}{Пример вывода}
\newcommand{\sampleexplanation}[1]{Объяснение примера #1}
\newcommand{\sampleexplanations}{Объяснение примеров}
\newcommand{\timelimit}{Ограничение по времени}
\newcommand{\memorylimit}{Ограничение на память}
\newcommand{\seconds}{сек}
\newcommand{\megabytes}{MB}
\newcommand{\group}{Группа}
\newcommand{\points}{Очки}
\newcommand{\limitsname}{Ограничения}
\newcommand{\additionalconstraints}{Дополнительные ограничения}
\newcommand{\testgroups}{
Тесты разделены на группы. Очки за группу даются только если корректно решены все тесты в группе.
}
\fi
\def\version{jury-1}
\problemname{Генетика}
Клонирование себя -- способ маскировки, которым часто пользуются ужасные злодеи, собирающиеся захватить вселенную.
Вы недавно захватили ужасного злодея вместе с его $N-1$ клонами, и пытаетесь узнать, кто же из них всех является оригинальным злодеем, а кто -- клоны.

К счастью, у вас есть доступ к последовательностям ДНК всех подозреваемых. Каждая последовательность состоит из 
$M$ символов \texttt{A}, \texttt{C}, \texttt{G} или \texttt{T}.
Известно, что клоны сделаны неидеально -- их ДНК отличается от оригинала ровно в $K$ позициях.

Можете ли вы распознать настоящего злодея?

\section*{\inputsection}
На первой строке даны три целых числа $N$, $M$, и $K$, где $1 \le K \le M$.
На следующих $N$ строках даны последовательности ДНК.
Каждая из них состоит из $M$ символов \texttt{A}, \texttt{C}, \texttt{G} или \texttt{T}.

Только одна из данных в файле последовательностей отличается от каждой из остальных ровно в $K$ позициях.

Внимание: объем входных данных в этой задаче довольно большой и требует использования быстрых библиотек ввода-вывода в Java.

\section*{\outputsection}
Выведите целое число -- номер ДНК последовательности настоящего злодея. Последовательности нумеруются с $1$.

\section*{\constraints}
\testgroups

\noindent
\begin{tabular}{| l | l | l | l |}
\hline
  \group & \points & \limitsname & \additionalconstraints \\ \hline
  1      & 27      & $3 \le N, M \le 100$ & \\ \hline
  2      & 19      & $3 \le N, M \le 1800$ & Только символы \texttt{A} и \texttt{C}. \\ \hline
  3      & 28      & $3 \le N, M \le 4100$ & Только символы \texttt{A} и \texttt{C}. \\ \hline
  4      & 26      & $3 \le N, M \le 4100$ & \\ \hline
\end{tabular}
