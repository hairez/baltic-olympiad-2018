\def\version{jury-draft}
\problemname{Genetics}
A common way for villains that mean to take over the world to protect
themselves is to clone themselves a number of times to avoid getting caught.
You have managed to catch an evil villain and her $N-1$ clones, and are now
trying to figure out which one of them is the real villain.

To your aid you have from each person a DNA sequence, simplified to consist of
$M$ ones and zeroes. You also know that the clones are not perfectly made;
rather, their sequences differ in exactly $K$ places compared to the real
villain's. For each clone there is also some other clone whose sequence differs
from it on either more or fewer than $K$ places.

Can you find the real villain?

\section*{\inputsection}
The first line contains three integers $N, M, K$ ($2 \le N, M \le 4100, 1 \le K \le M$).

Then follows $N$ lines, representing the DNA sequences.
Each of these lines consists of $M$ characters, all either \texttt{0} or \texttt{1}.

\section*{\outputsection}
Output an integer -- the index of the DNA sequence that belongs to the villain.
The sequences are numbered starting from $1$.

\section*{\constraints}
Your solution will be tested on a set of test groups.
To get points for a group you need to solve all test cases in the group.

\noindent
\begin{tabular}{| l | l | l |}
\hline
\group & \points & \limitsname \\ \hline
1     & 20     & $2 \le N, M \le 100$ \\ \hline
2     & 30     & $2 \le N, M \le 1500$ \\ \hline
3     & 50     & $2 \le N, M \le 4100$ \\ \hline
\end{tabular}
