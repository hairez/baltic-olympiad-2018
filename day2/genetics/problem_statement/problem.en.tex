\ifx\boi\undefined\ifx\problemname\undefined
\providecommand\sampleinputname{}
\providecommand\sampleoutputname{}
\documentclass[english]{templates/boi}
\problemlanguage{.en}
\fi
\newcommand{\boi}{Baltic Olympiad in Informatics}
\newcommand{\practicesession}{Practice Session}
\newcommand{\contestdates}{April 27 - May 1, 2018}
\newcommand{\dayone}{Day 1}
\newcommand{\daytwo}{Day 2}
\newcommand{\licensingtext}{This problem is licensed under CC BY-SA 4.0.}
\newcommand{\problem}{Problem}
\newcommand{\inputsection}{Input}
\newcommand{\outputsection}{Output}
\newcommand{\interactivity}{Interactivity}
\newcommand{\grading}{Grading}
\newcommand{\scoring}{Scoring}
\newcommand{\constraints}{Constraints}
\renewcommand{\sampleinputname}{Sample Input}
\renewcommand{\sampleoutputname}{Sample Output}
\newcommand{\sampleexplanation}[1]{Explanation of Sample #1}
\newcommand{\sampleexplanations}{Explanation of Samples}
\newcommand{\timelimit}{Time Limit}
\newcommand{\memorylimit}{Memory Limit}
\newcommand{\seconds}{s}
\newcommand{\megabytes}{MB}
\newcommand{\group}{Group}
\newcommand{\points}{Points}
\newcommand{\limitsname}{Limits}
\newcommand{\additionalconstraints}{Additional Constraints}
\newcommand{\testgroups}{
Your solution will be tested on a set of test groups, each worth a number of points.
Each test group contains a set of test cases.
To get the points for a test group you need to solve all test cases in the test group.
Your final score will be the maximum score of a single submission.
}
\fi
\def\version{jury-1}
\problemname{Genetics}
For villains that intend to take over the world, a common way to avoid getting caught
is to clone themselves. You have managed to catch an evil villain and her $N-1$ clones,
and you are now trying to figure out which one of them is the real villain.

To your aid you have each person's DNA sequence, consisting of $M$ characters, each being either
\texttt{A}, \texttt{C}, \texttt{G} or \texttt{T}.
You also know that the clones are not perfectly made;
rather, their sequences differ in exactly $K$ places compared to the real villain's.

Can you identify the real villain?

\section*{\inputsection}
The first line contains the three integers $N$, $M$, and $K$, where $1 \le K \le M$.
The following $N$ lines represent the DNA sequences.
Each of these lines consists of $M$ characters, each of which is either \texttt{A}, \texttt{C}, \texttt{G} or \texttt{T}.

In the input, there is exactly one sequence that differs from all the other sequences in exactly $K$ places.

Warning: this problem has rather large amounts of input, and will require fast IO in Java.

\section*{\outputsection}
Output an integer -- the index of the DNA sequence that belongs to the villain.
The sequences are numbered starting from $1$.

\section*{\constraints}
\testgroups

\noindent
\begin{tabular}{| l | l | l | l |}
\hline
  \group & \points & \limitsname & \additionalconstraints \\ \hline
  1      & 27      & $3 \le N, M \le 100$ & \\ \hline
  2      & 19      & $3 \le N, M \le 1800$ & All characters are either \texttt{A} or \texttt{C}. \\ \hline
  3      & 28      & $3 \le N, M \le 4100$ & All characters are either \texttt{A} or \texttt{C}. \\ \hline
  4      & 26      & $3 \le N, M \le 4100$ & \\ \hline
\end{tabular}
