\ifx\boi\undefined\ifx\problemname\undefined
\providecommand\sampleinputname{}
\providecommand\sampleoutputname{}
\documentclass[norsk]{templates/boi}
\problemlanguage{.no}
\fi
\newcommand{\boi}{Baltic Olympiad in Informatics}
\newcommand{\practicesession}{Practice Session}
\newcommand{\contestdates}{April 27 - May 1, 2018}
\newcommand{\dayone}{Dag 1}
\newcommand{\daytwo}{Dag 2}
\newcommand{\licensingtext}{This problem is licensed under CC BY-SA 4.0.}
\newcommand{\problem}{Problem}
\newcommand{\inputsection}{Input}
\newcommand{\outputsection}{Output}
\newcommand{\interactivity}{Interaktivitet}
\newcommand{\grading}{Grading}
\newcommand{\scoring}{Scoring}
\newcommand{\constraints}{Begrensninger}
\renewcommand{\sampleinputname}{Sample Input}
\renewcommand{\sampleoutputname}{Sample Output}
\newcommand{\sampleexplanation}[1]{Forklaring av eksempel #1}
\newcommand{\sampleexplanations}{Eksempelforklaring}
\newcommand{\timelimit}{Time Limit}
\newcommand{\memorylimit}{Memory Limit}
\newcommand{\seconds}{s}
\newcommand{\megabytes}{MB}
\newcommand{\group}{Group}
\newcommand{\points}{Points}
\newcommand{\limitsname}{Limits}
\newcommand{\additionalconstraints}{Yttligere begrensninger}
\newcommand{\testgroups}{
Løsningen din vil bli testet på et sett av testgrupper, hver verdt et visst antall poeng.
Hver testgruppe inneholder et sett av tester.
For å få poeng for en testgruppe må du løse alle testene i den gruppen.
Din endelige poengsum vil være den høyeste poengsummen du har fått på en enkelt innlevering.
}
\fi
\def\version{jury-1}
\problemname{Genetikk}
En vanlig teknikk som superskurker bruker for å unngå å bli fanget er å klone seg selv.
Du har klart å fange en ond superskurk og hennes $N-1$ kloner, men du ønsker å finne ut hvem
av de som er den ekte superskurken.

Til hjelp har du hver persons DNA-sekvens, som hver består av $M$ tegn, hvor hvert tegn er enten
\texttt{A}, \texttt{C}, \texttt{G} eller \texttt{T}. Du vet også at klonene aldri er helt perfekte kopier,
men at de har DNA-sekvenser som skiller seg fra den ekte superskurken på nøyaktig $K$ posisjoner.

Kan du identifisere den ekte superskurken?

\section*{\inputsection}
Første linje inneholder tre heltall $N$, $M$, og $K$, hvor $1 \le K \le M$.
De neste $N$ linjene representerer DNA sekvenser.
Hver av disse linjene inneholder $M$ tegn, hvor hvert tegn er enten \texttt{A}, \texttt{C}, \texttt{G} eller \texttt{T}.

I input vil det være nøyaktig én sekvens som skiller seg fra alle de andre sekvensene i nøyaktig $K$ posisjoner.

Advarsel: dette problemet har ganske mye input og vil kreve at du bruker raske IO-rutiner om du bruker Java.

\section*{\outputsection}
Skriv ut et heltall -- indeksen av DNA-sekvensen som tilhører superskurken.
Sekvensene er numerert begynnende med $1$.

\section*{\constraints}
\testgroups

\noindent
\begin{tabular}{| l | l | l | l |}
\hline
  \group & \points & \limitsname & \additionalconstraints \\ \hline
  1      & 27      & $3 \le N, M \le 100$ & \\ \hline
  2      & 19      & $3 \le N, M \le 1800$ & Hvert tegn er enten \texttt{A} eller \texttt{C}. \\ \hline
  3      & 28      & $3 \le N, M \le 4100$ & Hvert tegn er enten \texttt{A} eller \texttt{C}. \\ \hline
  4      & 26      & $3 \le N, M \le 4100$ & \\ \hline
\end{tabular}
