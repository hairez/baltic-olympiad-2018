\ifx\boi\undefined\ifx\problemname\undefined
\providecommand\sampleinputname{}
\providecommand\sampleoutputname{}
\documentclass[english]{templates/boi}
\problemlanguage{.en}
\fi
\newcommand{\boi}{Baltic Olympiad in Informatics}
\newcommand{\practicesession}{Practice Session}
\newcommand{\contestdates}{April 27 - May 1, 2018}
\newcommand{\dayone}{Day 1}
\newcommand{\daytwo}{Day 2}
\newcommand{\licensingtext}{This problem is licensed under CC BY-SA 4.0.}
\newcommand{\problem}{Problem}
\newcommand{\inputsection}{Input}
\newcommand{\outputsection}{Output}
\newcommand{\interactivity}{Interactivity}
\newcommand{\grading}{Grading}
\newcommand{\scoring}{Scoring}
\newcommand{\constraints}{Constraints}
\renewcommand{\sampleinputname}{Sample Input}
\renewcommand{\sampleoutputname}{Sample Output}
\newcommand{\sampleexplanation}[1]{Explanation of Sample #1}
\newcommand{\sampleexplanations}{Explanation of Samples}
\newcommand{\timelimit}{Time Limit}
\newcommand{\memorylimit}{Memory Limit}
\newcommand{\seconds}{s}
\newcommand{\megabytes}{MB}
\newcommand{\group}{Group}
\newcommand{\points}{Points}
\newcommand{\limitsname}{Limits}
\newcommand{\additionalconstraints}{Additional Constraints}
\newcommand{\testgroups}{
Your solution will be tested on a set of test groups, each worth a number of points.
Each test group contains a set of test cases.
To get the points for a test group you need to solve all test cases in the test group.
Your final score will be the maximum score of a single submission.
}
\fi
\def\version{jury-1}
\problemname{Genetyka}

Złoczyńcy, którzy dążą do podboju świata oczywiście nie chcą być złapani. Znaną
i powszechnie stosowaną metodą, aby tego dokonać jest klonowanie samego siebie.
Udało Ci się złapać złoczyńcę oraz jego $N-1$ klonów i teraz próbujesz wydedukować,
który z nich jest prawdziwym złoczyńcą.

Łańcuch DNA każdej złapanej osoby składa się z $M$ znaków, każdy z nich to
\texttt{A}, \texttt{C}, \texttt{G} lub \texttt{T}. Wiesz także, że klony nie są
perfekcyjne; w szczególności ich sekwencje DNA różnią się w dokładnie $K$ miejscach
porównując z prawdziwym złoczyńcą.

Czy potrafisz zidentyfikować prawdziwego złoczyńcę?

\section*{\inputsection}
W pierwszym wierszu dane są trzy liczby całkowite $N$, $M$ oraz $K$, gdzie $1 \le K \le M$.
Każdy z kolejnych $N$ wierszy zawiera sekwencję DNA.
Każdy z tych wierszy składa się z $M$ znaków, każdy z nich to \texttt{A}, \texttt{C}, \texttt{G} lub \texttt{T}.

Wśród sekwencji z wejścia jest dokładnie jedna sekwencja, która różni się od pozostałych na dokładnie $K$ pozycjach.

\section*{\outputsection}
Wypisz jedną liczbę całkowitą -- indeks sekwencji DNA, który należy do prawdziwego złoczyńcy.
Sekwencje są ponumerowane, poczynając od $1$.

\section*{\constraints}
\testgroups

\noindent
\begin{tabular}{| l | l | l | l |}
\hline
  \group & \points & \limitsname & \additionalconstraints \\ \hline
  1      & 27      & $2 \le N, M \le 100$ & \\ \hline
  2      & 19      & $2 \le N, M \le 1600$ & Wszystkie znaki to \texttt{A} lub \texttt{C}. \\ \hline
  3      & 28      & $2 \le N, M \le 4100$ & Wszystkie znaki to \texttt{A} lub \texttt{C}. \\ \hline
  4      & 26      & $2 \le N, M \le 4100$ & \\ \hline
\end{tabular}
