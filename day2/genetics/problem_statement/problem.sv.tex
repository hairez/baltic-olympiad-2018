\ifx\boi\undefined\ifx\problemname\undefined
\providecommand\sampleinputname{}
\providecommand\sampleoutputname{}
\documentclass[english]{templates/boi}
\problemlanguage{.en}
\fi
\newcommand{\boi}{Baltic Olympiad in Informatics}
\newcommand{\practicesession}{Practice Session}
\newcommand{\contestdates}{April 27 - May 1, 2018}
\newcommand{\dayone}{Day 1}
\newcommand{\daytwo}{Day 2}
\newcommand{\licensingtext}{This problem is licensed under CC BY-SA 4.0.}
\newcommand{\problem}{Problem}
\newcommand{\inputsection}{Input}
\newcommand{\outputsection}{Output}
\newcommand{\interactivity}{Interactivity}
\newcommand{\grading}{Grading}
\newcommand{\scoring}{Scoring}
\newcommand{\constraints}{Constraints}
\renewcommand{\sampleinputname}{Sample Input}
\renewcommand{\sampleoutputname}{Sample Output}
\newcommand{\sampleexplanation}[1]{Explanation of Sample #1}
\newcommand{\sampleexplanations}{Explanation of Samples}
\newcommand{\timelimit}{Time Limit}
\newcommand{\memorylimit}{Memory Limit}
\newcommand{\seconds}{s}
\newcommand{\megabytes}{MB}
\newcommand{\group}{Group}
\newcommand{\points}{Points}
\newcommand{\limitsname}{Limits}
\newcommand{\additionalconstraints}{Additional Constraints}
\newcommand{\testgroups}{
Your solution will be tested on a set of test groups, each worth a number of points.
Each test group contains a set of test cases.
To get the points for a test group you need to solve all test cases in the test group.
Your final score will be the maximum score of a single submission.
}
\fi
\def\version{jury-1}
\problemname{Genetik}
Vissa förbrytare vill ta över världen. Ett vanligt sätt för förbrytare 
att undvika att bli straffade är att klona sig själva. Du har nu fångat en förbrytare och
hennes $N-1$ kloner. Du ska ta reda på vem som är den riktiga förbrytaren.

Du har tillgång till varje infångad persons DNA, som alla består av $M$ bokstäver där varje bokstav är antingen \texttt{A}, \texttt{C}, \texttt{G} eller \texttt{T}.
Du vet också att klonerna inte är perfekta kloner, utan istället skiljer sig 
DNA-sekvensen för varje klon på exakt $K$ ställen jämfört med den riktiga förbrytaren.

Kan du identifiera den riktiga förbrytaren?

\section*{\inputsection}
Den första raden innehåller tre heltal $N$, $M$ och $K$, där $1 \le K \le M$.
Följande $N$ rader innehåller DNA-sekvenser.
Varje av dessa innehåller $M$ bokstäver, där varje bokstav är någon av \texttt{A}, \texttt{C}, \texttt{G} or \texttt{T}.

Det finns, i inputen, exakt en sekvens som skiljer sig från alla andra på exakt $K$ ställen.

\section*{\outputsection}

Skriv ut ett heltal -- indexet av DNA-sekvensen som tillhör förbrytaren.

Sekvensernas index börjar med $1$.

\section*{\constraints}
\testgroups

\noindent
\begin{tabular}{| l | l | l | l |}
\hline
  \group & \points & \limitsname & \additionalconstraints \\ \hline
  1      & 27      & $2 \le N, M \le 100$ & \\ \hline
  2      & 19      & $2 \le N, M \le 1600$ & Alla bokstäver är antingen \texttt{A} eller \texttt{C}. \\ \hline
  3      & 28      & $2 \le N, M \le 4100$ & Alla bokstäver är antingen \texttt{A} eller \texttt{C}. \\ \hline
  4      & 26      & $2 \le N, M \le 4100$ & \\ \hline
\end{tabular}
