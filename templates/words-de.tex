\ifx\problemname\undefined
\providecommand\sampleinputname{}
\providecommand\sampleoutputname{}
\documentclass[german]{templates/boi}
\problemlanguage{.de}
\fi
\newcommand{\boi}{Baltische Informatikolympiade}
\newcommand{\practicesession}{Practice Session}
\newcommand{\contestdates}{27. April - 1. Mai, 2018}
\newcommand{\dayone}{Tag 1}
\newcommand{\daytwo}{Tag 2}
\newcommand{\licensingtext}{Diese Aufgabenstellung ist unter der CC BY-SA 4.0 lizensiert.}
\newcommand{\problem}{Aufgabe}
\newcommand{\inputsection}{Eingabe}
\newcommand{\outputsection}{Ausgabe}
\newcommand{\interactivity}{Interaktivität}
\newcommand{\grading}{Grading}
\newcommand{\scoring}{Scoring}
\newcommand{\constraints}{Beschränkungen}
\renewcommand{\sampleinputname}{Beispieleingabe}
\renewcommand{\sampleoutputname}{Beispielausgabe}
\newcommand{\sampleexplanation}[1]{Beschreibung zu Beispiel #1}
\newcommand{\sampleexplanations}{Beschreibung der Beispiele}
\newcommand{\timelimit}{Zeitlimit}
\newcommand{\memorylimit}{Speicherlimit}
\newcommand{\seconds}{s}
\newcommand{\megabytes}{MB}
\newcommand{\group}{Gruppe}
\newcommand{\points}{Punkte}
\newcommand{\limitsname}{Limits}
\newcommand{\additionalconstraints}{Zusätzliche Beschränkungen}
\newcommand{\testgroups}{
Deine Lösung wird auf mehreren Testgruppen ausgeführt werden, von denen jede eine bestimmte Punktzahl wert ist.
Jede Testgruppe enthält mehrere Testcases. Um Punkte für eine Testgruppe zu bekommen, müssen alle Testfälle in der entsprechenden Gruppe gelöst werden.
Deine finale Punktzahl wird die maximal erreichte Punktzahl in einer einzelnen Einsendungen sein.
}
