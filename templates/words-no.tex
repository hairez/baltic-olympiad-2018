\ifx\problemname\undefined
\providecommand\sampleinputname{}
\providecommand\sampleoutputname{}
\documentclass[norsk]{templates/boi}
\problemlanguage{.no}
\fi
\newcommand{\boi}{Baltic Olympiad in Informatics}
\newcommand{\practicesession}{Practice Session}
\newcommand{\contestdates}{April 27 - May 1, 2018}
\newcommand{\dayone}{Dag 1}
\newcommand{\daytwo}{Dag 2}
\newcommand{\licensingtext}{This problem is licensed under CC BY-SA 4.0.}
\newcommand{\problem}{Problem}
\newcommand{\inputsection}{Input}
\newcommand{\outputsection}{Output}
\newcommand{\interactivity}{Interaktivitet}
\newcommand{\grading}{Grading}
\newcommand{\scoring}{Scoring}
\newcommand{\constraints}{Begrensninger}
\renewcommand{\sampleinputname}{Sample Input}
\renewcommand{\sampleoutputname}{Sample Output}
\newcommand{\sampleexplanation}[1]{Forklaring av eksempel #1}
\newcommand{\sampleexplanations}{Eksempelforklaring}
\newcommand{\timelimit}{Time Limit}
\newcommand{\memorylimit}{Memory Limit}
\newcommand{\seconds}{s}
\newcommand{\megabytes}{MB}
\newcommand{\group}{Group}
\newcommand{\points}{Points}
\newcommand{\limitsname}{Limits}
\newcommand{\additionalconstraints}{Yttligere begrensninger}
\newcommand{\testgroups}{
Løsningen din vil bli testet på et sett av testgrupper, hver verdt et visst antall poeng.
Hver testgruppe inneholder et sett av tester.
For å få poeng for en testgruppe må du løse alle testene i den gruppen.
Din endelige poengsum vil være den høyeste poengsummen du har fått på en enkelt innlevering.
}
