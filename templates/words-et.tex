\ifx\problemname\undefined
\providecommand\sampleinputname{}
\providecommand\sampleoutputname{}
\documentclass[estonian]{templates/boi}
\problemlanguage{.et}
\fi
\newcommand{\boi}{Balti Informaatikaolümpiaad}
\newcommand{\contestdates}{27. aprill -- 1. mai 2018}
\newcommand{\practicesession}{proovivoor}
\newcommand{\dayone}{1. võistluspäev}
\newcommand{\daytwo}{2. võistluspäev}
\newcommand{\licensingtext}{Avaldatud CC BY-SA 4.0 litsentsi all.}
\newcommand{\problem}{Ülesanne}
\newcommand{\inputsection}{Sisend}
\newcommand{\outputsection}{Väljund}
\newcommand{\interactivity}{Interaktsioon}
\newcommand{\grading}{Testimine}
\newcommand{\scoring}{Hindamine}
\newcommand{\constraints}{Piirangud}
\renewcommand{\sampleinputname}{Sisendi näide}
\renewcommand{\sampleoutputname}{Väljundi näide}
\newcommand{\sampleexplanation}[1]{Näite #1 selgitus}
\newcommand{\sampleexplanations}{Näidete selgitus}
\newcommand{\timelimit}{Ajalimiit}
\newcommand{\memorylimit}{Mälulimiit}
\newcommand{\seconds}{sek}
\newcommand{\megabytes}{MB}
\newcommand{\group}{Grupp}
\newcommand{\points}{Punkte}
\newcommand{\limitsname}{Piirangud}
\newcommand{\additionalconstraints}{Lisapiirangud}
\newcommand{\testgroups}{
Selles ülesandes on testid jagatud gruppidesse.
Iga grupi eest saavad punkte ainult need programmid, mis lahendavad õigesti kõik gruppi kuuluvad testid.
Sinu lõplik skoor on esitatud lahenduste skooride maksimum.
}
\renewcommand{\le}{\leqslant}
