\ifx\problemname\undefined
\providecommand\sampleinputname{}
\providecommand\sampleoutputname{}
\documentclass[swedish]{templates/boi}
\problemlanguage{.sv}
\fi
\newcommand{\practicesession}{Övningssession}
\newcommand{\dayone}{Dag 1}
\newcommand{\daytwo}{Dag 2}
\newcommand{\boi}{Baltic Olympiad in Informatics}
\newcommand{\contestdates}{27 april - 1 maj, 2018}
\newcommand{\licensingtext}{Detta problem är licensierat enligt CC BY-SA 4.0.}
\newcommand{\problem}{Problem}
\newcommand{\inputsection}{Indata}
\newcommand{\outputsection}{Utdata}
\newcommand{\interactivity}{Interaktivitet}
\newcommand{\grading}{Bedömning}
\newcommand{\scoring}{Poängsättning}
\newcommand{\constraints}{Begränsningar}
\renewcommand{\sampleinputname}{Exempel-indata}
\renewcommand{\sampleoutputname}{Exempel-utdata}
\newcommand{\sampleexplanation}[1]{Förklaring till Exempel #1}
\newcommand{\sampleexplanations}{Förklaring till exemplen}
\newcommand{\timelimit}{Tidsgräns}
\newcommand{\memorylimit}{Minnesgräns}
\newcommand{\seconds}{s}
\newcommand{\megabytes}{MB}
\newcommand{\group}{Grupp}
\newcommand{\points}{Poäng}
\newcommand{\limitsname}{Gränser}
\newcommand{\additionalconstraints}{Övriga begränsningar}
\newcommand{\testgroups}{
Din lösning kommer att testas på en uppsättning testgrupper, var och en värd en viss poäng.
Varje testgrupp innehåller flera testfall. 
För att få poäng för en testgrupp måste du klara alla testfallen i gruppen.
Din slutgiltiga poäng på problemet kommer att vara den maximala poängen av en enda submission.
}

