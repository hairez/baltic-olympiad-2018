\ifx\problemname\undefined
\providecommand\sampleinputname{}
\providecommand\sampleoutputname{}
\documentclass[lithuanian]{templates/boi}
\usepackage[L7x]{fontenc}
\problemlanguage{.lt}
\fi
\newcommand{\boi}{Baltijos šalių informatikos olimpiada}
\newcommand{\practicesession}{Bandomasis turas}
\newcommand{\contestdates}{2018 m. balandžio 27 - gegužės 1 d.}
\newcommand{\dayone}{1 diena}
\newcommand{\daytwo}{2 diena}
\newcommand{\licensingtext}{This problem is licensed under CC BY-SA 4.0.}
\newcommand{\problem}{Uždavinys}
\newcommand{\inputsection}{Pradiniai duomenys}
\newcommand{\outputsection}{Rezultatai}
\newcommand{\interactivity}{Realizacija}
\newcommand{\grading}{Vertinimas(?)}
\newcommand{\scoring}{Taškų skyrimas(?)}
\newcommand{\constraints}{Ribojimai}
\renewcommand{\sampleinputname}{Pradiniai duomenys}
\renewcommand{\sampleoutputname}{Rezultatai}
\newcommand{\sampleexplanation}[1]{Pirmojo pavyzdžio paaiškinimas}
\newcommand{\sampleexplanations}{Pavyzdžių paaiškinimai}
\newcommand{\timelimit}{Laiko ribojimas}
\newcommand{\memorylimit}{Atminties ribojimas}
\newcommand{\seconds}{s}
\newcommand{\megabytes}{MB}
\newcommand{\group}{Grupė}
\newcommand{\points}{Taškai}
\newcommand{\limitsname}{Ribojimai}
\newcommand{\additionalconstraints}{Papildomi ribojimai}
\newcommand{\testgroups}{
Jūsų sprendimas bus testuojamas su keliomis testų grupėmis, kiekviena kurių vertinama tam tikru skaičiumi taškų.
Kiekvieną testų grupę sudarys keletas testų.
Taškai už testų grupę skiriami tik jei įveikiate visus tos grupės testus.
}
