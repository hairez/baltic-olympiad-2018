\ifx\problemname\undefined
\providecommand\sampleinputname{}
\providecommand\sampleoutputname{}
\documentclass[nil]{templates/boi}
\ifdefined\babelprovide
  \babelprovide[import=lv,main]{latvian}
\fi
\problemlanguage{.lv}
\fi
\newcommand{\boi}{Baltijas informātikas olimpiāde}
\newcommand{\practicesession}{Izmēģinājuma kārta}
\newcommand{\contestdates}{27.~aprīlis~-- 1.~maijs, 2018}
\newcommand{\dayone}{1.~diena}
\newcommand{\daytwo}{2.~diena}
\newcommand{\licensingtext}{Šis uzdevums ir licencēts zem CC BY-SA~4.0.}
\newcommand{\problem}{Uzdevums}
\newcommand{\inputsection}{Ievaddati}
\newcommand{\outputsection}{Izvaddati}
\newcommand{\interactivity}{Komunikācija}
\newcommand{\grading}{Testēšana}
\newcommand{\scoring}{Vērtēšana}
\newcommand{\constraints}{Ierobežojumi}
\renewcommand{\sampleinputname}{Ievaddatu paraugs}
\renewcommand{\sampleoutputname}{Izvaddatu paraugs}
\newcommand{\sampleexplanation}[1]{#1.~parauga paskaidrojums}
\newcommand{\sampleexplanations}{Paraugu paskaidrojumi}
\newcommand{\timelimit}{Laika ierobežojums}
\newcommand{\memorylimit}{Atmiņas ierobežojums}
\newcommand{\seconds}{s}
\newcommand{\megabytes}{MB}
\newcommand{\group}{Grupa}
\newcommand{\points}{Punkti}
\newcommand{\limitsname}{Ierobežojumi}
\newcommand{\additionalconstraints}{Papildu ierobežojumi}
\newcommand{\testgroups}{%
Jūsu risinājums tiks testēts uz vairākām testu grupām, par katru no tām var iegūt punktus.
Katra testu grupa satur vienu vai vairākus testus.
Lai iegūtu punktus par testu grupu, jums ir pareizi jāatrisina visi testi šajā grupā.
Jūsu beigu vērtējums par uzdevumu būs starp visiem iesūtījumiem lielākais.%
}
