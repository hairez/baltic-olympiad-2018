\ifx\boi\undefined\ifx\problemname\undefined
\providecommand\sampleinputname{}
\providecommand\sampleoutputname{}
\documentclass[russian]{templates/boi}
\problemlanguage{.ru}
\fi
\newcommand{\boi}{Балтийская Олимпиада по Информатике}
\newcommand{\practicesession}{Тренировочный раунд}
\newcommand{\contestdates}{27 апреля - 1 мая, 2018}
\newcommand{\dayone}{День 1}
\newcommand{\daytwo}{День 2}
\newcommand{\licensingtext}{Задача публикуется под лицензией CC BY-SA 4.0.}
\newcommand{\problem}{Задача}
\newcommand{\inputsection}{Ввод}
\newcommand{\outputsection}{Вывод}
\newcommand{\interactivity}{Интерактивность}
\newcommand{\grading}{Оценивание}
\newcommand{\scoring}{Очки}
\newcommand{\constraints}{Ограничения}
\renewcommand{\sampleinputname}{Пример ввода}
\renewcommand{\sampleoutputname}{Пример вывода}
\newcommand{\sampleexplanation}[1]{Объяснение примера #1}
\newcommand{\sampleexplanations}{Объяснение примеров}
\newcommand{\timelimit}{Ограничение по времени}
\newcommand{\memorylimit}{Ограничение на память}
\newcommand{\seconds}{сек}
\newcommand{\megabytes}{MB}
\newcommand{\group}{Группа}
\newcommand{\points}{Очки}
\newcommand{\limitsname}{Ограничения}
\newcommand{\additionalconstraints}{Дополнительные ограничения}
\newcommand{\testgroups}{
Тесты разделены на группы. Очки за группу даются только если корректно решены все тесты в группе.
}
\fi
\def\version{jury-1}
\problemname{Ссылки}
\illustration{.40}{oldbooks}{\href{https://pixabay.com/en/books-old-books-antiquariat-read-1215672/}{CC0, congerdesign via Pixabay.}}

Грейс собирается прочитать некую научную книгу. Она привыкла всегда читать все книги-первоисточники, на которые ссылается изначальная книга, а 
затем на первоисточники, на которые ссылались они, и так далее. Из-за этого она обычно в итоге читает очень много книг вдобавок к той, которую хотела прочесть изначально.

Библиотекари постоянно жалуются на то, что Грейс берет в библиотеке слишком много книг. Чтобы их поменьше расстраивать, Грейс хочет
найти порядок прочтения книг, при котором суммарное время удержания книг дома было бы наименьшим.

Всего Грейс необходимо прочитать $N$ разных книг. Книги пронумерованы от $1$ до $N$, а на прочтение книги $i$ уходит $K_i$ минут.
Каждая книга $i$ содержит список ссылок, состоящий из $F_i$ книг.
Книга, которую Грейс собиралась прочитать изначально, имеет номер $1$.
Прежде чем начать читать книгу номер $1$, Грейс взяла из библиотеки все $N$ книг на дом.

Процесс прочтения каждой книги у Грейс проходит следующим образом:

\begin{itemize}
\item Сначала она открывает книгу и сразу же читает список ссылок. Это занимает у неё одну минуту.
\item Затем она прочитывает все книги в списке ссылок, в любом порядке на её усмотрение.
\item После этого она читает саму книгу, и возвращает её в библиотеку. Это занимает ещё $K_i$ минут.
\end{itemize}

Вам необходимо найти наименьшую возможную сумму времен хранения всех книг дома, при условии что Грейс будет читать их в оптимальном порядке.
Книги могут иметь пустые списки ссылок. Каждая книга, кроме книги номер $1$ {\em присутствует ровно в одном списке ссылок}.
Также известно, что ссылки не образуют циклов.

\section*{\inputsection}
На первой строке ввода дано целое число $N$ ($1 \le N \le 100\,000$).
Далее следут $N$ строк, по одной на каждую книгу от $1$ до $N$.
На каждой из этих строк дано целое число $K_i$ ($1 \le K_i \le 1\,000$) -- количество минут, необходимое для прочтения книги.
Затем дано целое число $F_i$ ($0 \le F_i < N$), за которым следуют $F_i$ целых чисел -- номера книг, на которые ссылается книга $i$.

\section*{\outputsection}
Вывести одно целое число -- минимальное суммарное время удержания всех книг.

\section*{\constraints}
\testgroups

\noindent
\begin{tabular}{| l | l | l |}
\hline
\group & \points & \limitsname \\ \hline
1     & 20     & $1 \le N \le 10, 1 \le K_i \le 1000$ \\ \hline
2     & 30     & $1 \le N \le 50, 1 \le K_i \le 10$, $F_i \le 5$ \\ \hline
3     & 20     & $1 \le N \le 100\,000, 1 \le K_i \le 1000, F_i \le 5$ \\ \hline
4     & 30     & $1 \le N \le 100\,000, 1 \le K_i \le 1000$ \\ \hline
\end{tabular}

\section*{\sampleexplanation{1}}
минута 1: открыть книгу 1 \\
минута 2: открыть книгу 2 \\
минута 3: открыть книгу 4 \\
минута 4: закрыть книгу 4 (время удержания -- 4 минуты) \\
минута 14: закрыть книгу 2 (время удержания -- 14 минут) \\
минута 15: открыть книгу 3 \\
минута 16: открыть книгу 5 \\
минута 17: закрыть книгу 5 (время удержания -- 17 минут) \\
минута 37: закрыть книгу 3 (время удержания -- 37 минут) \\
минута 38: закрыть книгу 1 (время удержания -- 38 минут) \\
