\ifx\boi\undefined\ifx\problemname\undefined
\providecommand\sampleinputname{}
\providecommand\sampleoutputname{}
\documentclass[icelandic]{templates/boi}
\problemlanguage{.is}
\fi
\newcommand{\boi}{Eystrasaltsólympíuleikarnir í forritnu}
\newcommand{\practicesession}{Æfingakeppni}
\newcommand{\contestdates}{27. apríl - 1. maí, 2018}
\newcommand{\dayone}{Dagur 1}
\newcommand{\daytwo}{Dagur 2}
\newcommand{\licensingtext}{This problem is licensed under CC BY-SA 4.0.}
\newcommand{\problem}{Verkefni}
\newcommand{\inputsection}{Inntak}
\newcommand{\outputsection}{Úttak}
\newcommand{\interactivity}{Gagnvirkni}
\newcommand{\grading}{Flokkun}
\newcommand{\scoring}{Stigagjöf}
\newcommand{\constraints}{Takmarkanir}
\renewcommand{\sampleinputname}{Sýnidæmis inntak}
\renewcommand{\sampleoutputname}{Sýnidæmis úttak}
\newcommand{\sampleexplanation}[1]{Útskýring á sýnidæmi #1}
\newcommand{\sampleexplanations}{Útskýringar á sýnidæmum}
\newcommand{\timelimit}{Tímamörk}
\newcommand{\memorylimit}{Minnismörk}
\newcommand{\seconds}{s}
\newcommand{\megabytes}{MB}
\newcommand{\group}{Hópur}
\newcommand{\points}{Stig}
\newcommand{\limitsname}{Takmarkanir}
\newcommand{\additionalconstraints}{Auka takmarkanir}
\newcommand{\testgroups}{
Lausnin þín verður prófuð á einhvern fjölda prufuhópa, hver hópur gefur einhvern fjölda stiga.
Hver hópur inniheldur einhvern fjölda prufutilvika.
Til að fá stig fyrir hóp þarftu að leysa öll prufutilvik innan hópsins.
Lokastigin eru fengin úr skilunum sem gáfu hæst stig.
}
\fi
\def\version{jury-1}
\problemname{Tilvitnanir}
\illustration{.40}{oldbooks}{\href{https://pixabay.com/en/books-old-books-antiquariat-read-1215672/}{CC0 Public Domain, congerdesign via Pixabay.}}

Grace ætlar að lesa sérstaka vísindabók.
Hún passar sig alltaf að lesa heimildirnar sem bækurnar vitna í, og einnig
heimildir heimildanna og svo framvegis.
Það endar venjulega þannig að hún les mikið meira en bara bókina sem hún ætlaði upprunalega að lesa.
Bókasafnsverðirnir eru sífellt að kvarta yfir hversu lengi Grace er með bækur í láni,
þess vegna vil hún finna hvaða röð hún þarf að lesa bækurnar í til að stytta útlánstímann eins og hún getur.

Það eru $N$ bækur sem hún mun þurfa að lesa.
Bækurnar eru númeraðar frá $1$ til $N$, og það tekur $K_i$ mínútur að lesa bók $i$.
Sérhver bók $i$ er með tilvitnanalista með $F_i$ bókum.
Bókin sem Grace vildi upprunalega lesa er númer $1$.
Grace hefur þegar tekið allar bækurnar að láni áður en hún byrjar á bók $1$.

Þegar Grace les bók gerir hún eftirfarandi:

\begin{itemize}
\item Fyrst opnar hún bókina og les tilvitnanalistann, sem tekur mínútu,
\item svo les hún allar bækurnar á listanum, í einhverri röð sem hún velur,
\item þar á eftir les hún bókina sjálfa og skilar henni á bókasafnið, sem tekur $K_i$ mínútur.
\end{itemize}

Núna viltu reikna út minnsta mögulega lánstíma fyrir allar bækur, gefið að Grace lesi þær í bestu röðinni.
Bækur geta innihaldið tóman tilvitnanalista, og allar bækur nema bók $1$ {\em mun koma fyrir í nákvæmlega einum tilvitnanalista}.
Það verða engar hringrásir af tilvitnunum.

\section*{\inputsection}
Fyrsta línan inniheldur heiltöluna $N$ ($1 \le N \le 100\,000$).
Svo fylgja $N$ línur, ein fyrir hverja bók $1$ til $N$.
Í hverri línu verður tala $K_i$ ($1 \le K_i \le 1\,000$), fjöldi mínútna sem tekur að lesa bókina,
svo fylgir tala $F_i$ ($0 \le F_i < N$), og $F_i$ tölur í viðbót, númer bókanna sem bók $i$ vitnar í. 

\section*{\outputsection}
Skrifaðu út eina jákvæða heiltölu, minnsta samanlagða lánstíma fyrir allar bækur.

\section*{\constraints}
\testgroups

\noindent
\begin{tabular}{| l | l | l |}
\hline
\group & \points & \limitsname \\ \hline
1     & 20     & $1 \le N \le 10, 1 \le K_i \le 1000$ \\ \hline
2     & 30     & $1 \le N \le 50, 1 \le K_i \le 10$, $F_i \le 5$ \\ \hline
3     & 20     & $1 \le N \le 100\,000, 1 \le K_i \le 1000, F_i \le 5$ \\ \hline
4     & 30     & $1 \le N \le 100\,000, 1 \le K_i \le 1000$ \\ \hline
\end{tabular}

\section*{\sampleexplanation{1}}
tími 1: opna bók 1 \\
tími 2: opna bók 2 \\
tími 3: opna bók 4 \\
tími 4: loka bók 4 (lánstími 4 mínútur) \\
tími 14: loka bók 2 (lánstími 14 mínútur) \\
tími 15: opna bók 3 \\
tími 16: opna bók 5 \\
tími 17: loka bók 5 (lánstími 17 mínútur) \\
tími 37: loka bók 3 (lánstími 37 mínútur) \\
tími 38: loka bók 1 (lánstími 38 mínútur) \\
