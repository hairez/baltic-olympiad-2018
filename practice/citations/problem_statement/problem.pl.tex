\ifx\boi\undefined\ifx\problemname\undefined
\providecommand\sampleinputname{}
\providecommand\sampleoutputname{}
\documentclass[polish]{templates/boi}
\problemlanguage{.pl}
\fi
\newcommand{\boi}{Bałtycka Olimpiada Informatyczna}
\newcommand{\practicesession}{Dzień próbny}
\newcommand{\contestdates}{27 kwietnia - 1 maja 2018}
\newcommand{\dayone}{Dzień 1}
\newcommand{\daytwo}{Dzień 2}
\newcommand{\licensingtext}{To zadanie jest objęte licencją CC BY-SA 4.0.}
\newcommand{\problem}{Zadanie}
\newcommand{\inputsection}{Wejście}
\newcommand{\outputsection}{Wyjście}
\newcommand{\interactivity}{Interakcja}
\newcommand{\grading}{Ocenianie}
\newcommand{\scoring}{Punkty}
\newcommand{\constraints}{Ograniczenia}
\renewcommand{\sampleinputname}{Przykładowe wejście}
\renewcommand{\sampleoutputname}{Przykładowe wyjście}
\newcommand{\sampleexplanation}[1]{Wyjaśnienie do przykładu #1}
\newcommand{\sampleexplanations}{Wyjaśnienie do przykładów}
\newcommand{\timelimit}{Limit czasu}
\newcommand{\memorylimit}{Limit pamięci}
\newcommand{\seconds}{s}
\newcommand{\megabytes}{MB}
\newcommand{\group}{Grupa}
\newcommand{\points}{Punkty}
\newcommand{\limitsname}{Limity}
\newcommand{\additionalconstraints}{Dodatkowe ograniczenia}
\newcommand{\testgroups}{
Zestaw testów dzieli się na kilka grup, każda jest warta pewną liczbę punktów.
Każda grupa składa się z jednego bądź większej liczby testów.
Aby otrzymać punkty za daną grupę, Twoje rozwiązanie musi przejść wszystkie testy z tej grupy.
Ostateczny wynik za zadanie jest liczony jako maksymalny wynik z pojedynczych zgłoszeń.
}



% \fi
% \newcommand{\boi}{Baltic Olympiad in Informatics}
% \newcommand{\practicesession}{Practice Session}
% \newcommand{\contestdates}{April 27 - May 1, 2018}
% \newcommand{\dayone}{Day 1}
% \newcommand{\daytwo}{Day 2}
% \newcommand{\licensingtext}{This problem is licensed under CC BY-SA 4.0.}
% \newcommand{\problem}{Problem}
% \newcommand{\inputsection}{Input}
% \newcommand{\outputsection}{Output}
% \newcommand{\interactivity}{Interactivity}
% \newcommand{\grading}{Grading}
% \newcommand{\scoring}{Scoring}
% \newcommand{\constraints}{Constraints}
% \renewcommand{\sampleinputname}{Sample Input}
% \renewcommand{\sampleoutputname}{Sample Output}
% \newcommand{\sampleexplanation}[1]{Explanation of Sample #1}
% \newcommand{\sampleexplanations}{Explanation of Samples}
% \newcommand{\timelimit}{Time Limit}
% \newcommand{\memorylimit}{Memory Limit}
% \newcommand{\seconds}{s}
% \newcommand{\megabytes}{MB}
% \newcommand{\group}{Group}
% \newcommand{\points}{Points}
% \newcommand{\limitsname}{Limits}
% \newcommand{\additionalconstraints}{Additional Constraints}
% \newcommand{\testgroups}{
% Your solution will be tested on a set of test groups, each worth a number of points.
% Each test group contains a set of test cases.
% To get the points for a test group you need to solve all test cases in the test group.
% }
\fi
\def\version{jury-1}
\problemname{Odniesienia}
\illustration{.40}{oldbooks}{\href{https://pixabay.com/en/books-old-books-antiquariat-read-1215672/}{CC0 Public Domain, congerdesign via Pixabay.}}

Gracjan przymierza się do przeczytania książki naukowej.
Ponieważ czyta bardzo uważnie, zawsze czyta również pozycje, do których odnosi się
książka. I pozycje, do których odnoszą się owe pozycje.
I tak dalej. Zwykle kończy się tak, że Gracjan czyta o wiele więcej książek
niż zamierzał.
Ponieważ bibliotekarka zaczyna narzekać na hobby Gracjana
skutkujące w długie wypożyczenia, zamierza on znaleźć taką kolejność
czytania książek, aby zminimalizować łączny czas ich wypożyczenia.

Gracjan będzie musiał przeczytać $N$ książek.
Książki są ponumerowane od $1$ do $N$, przeczytanie książki numer $i$ zajmuje $K_i$ minut.
Książka $i$ ma bibliografię (listę odniesień) zawierającą $F_i$ pozycji.
Początkowo Gracjan zamierza przeczytać książkę $1$.
Przed rozpoczęciem czytania Gracjan wypożyczył wszystkie $N$ książek.


Czytając książkę numer $i$ Gracjan:
\begin{itemize}
\item Otwiera książkę i czyta bibliografię w ciągu jednej minuty,
\item potem czyta wszystkie pozycje, do których odnosi się książka
\item i dopiero czyta książkę $i$, co zajmuje mu $K_i$ minut. Przeczytana książka jest natychmiast zwracana w 0 minut.
\end{itemize}

Twoim zadaniem jest policzenie minimalnego łącznego czasu wypożyczenia książek przy założeniu, że Gracjan czyta
je w optymalnej kolejności. Książka może zawierać pustą bibliografię i każda książka za wyjątkiem książki $1$
{\em jest wymieniona na w dokładnie jednej bibliografii}. Odniesienia nie tworzą cykli.

\section*{\inputsection}
W pierwszej linii wejścia znajduje się jedna liczba całkowita $N$ ($1 \le N \le 100\,000$).
Kolejne $N$ wierszy opisuje książki w kolejności od $1$ do $N$.
Na początku pojedynczego opisu znajdują się liczby całkowite $K_i$ i $F_i$ ($1 \le K_i \le 1\,000$, $0 \le F_i < N$)
oznaczające odpowiednio liczbę minut, przez które Gracjan czyta $i$-tą oraz liczbę pozycji w bibliografii.
Kolejne $F_i$ liczb zawiera numery książek, do których odnosi się $i$-ta książka.

\section*{\outputsection}
Na wyjście wypisz pojedynczą liczbę całkowitą -- minimalny łączny czas wypożyczenia wszystkich książek.

\section*{\constraints}
\testgroups

\noindent
\begin{tabular}{| l | l | l |}
\hline
\group & \points & \limitsname \\ \hline
1   & 20   & $1 \le N \le 10, 1 \le K_i \le 1000$ \\ \hline
2   & 30   & $1 \le N \le 50, 1 \le K_i \le 10$, $F_i \le 5$ \\ \hline
3   & 20   & $1 \le N \le 100\,000, 1 \le K_i \le 1000, F_i \le 5$ \\ \hline
4   & 30   & $1 \le N \le 100\,000, 1 \le K_i \le 1000$ \\ \hline
\end{tabular}

\section*{\sampleexplanation{1}}
min 1: otwarcie książki 1 \\
min 2: otwarcie książki 2 \\
min 3: otwarcie książki 4 \\
min 4: zamknięcie i zwrot książki 4 (czas wypożyczenia 4 minuty) \\
min 14: zamknięcie i zwrot książki 2 (czas wypożyczenia 14 minut) \\
min 15: otwarcie książki 3 \\
min 16: otwarcie książki 5 \\
min 17: zamknięcie i zwrot książki 5 (czas wypożyczenia 17 minut) \\
min 37: zamknięcie i zwrot książki 3 (czas wypożyczenia 37 minut) \\
min 38: zamknięcie i zwrot książki 1 (czas wypożyczenia 38 minut) \\
