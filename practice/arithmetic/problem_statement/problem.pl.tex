\ifx\boi\undefined\ifx\problemname\undefined
\providecommand\sampleinputname{}
\providecommand\sampleoutputname{}
\documentclass[polish]{templates/boi}
\problemlanguage{.pl}
\fi
\newcommand{\boi}{Bałtycka Olimpiada Informatyczna}
\newcommand{\practicesession}{Dzień próbny}
\newcommand{\contestdates}{27 kwietnia - 1 maja 2018}
\newcommand{\dayone}{Dzień 1}
\newcommand{\daytwo}{Dzień 2}
\newcommand{\licensingtext}{To zadanie jest objęte licencją CC BY-SA 4.0.}
\newcommand{\problem}{Zadanie}
\newcommand{\inputsection}{Wejście}
\newcommand{\outputsection}{Wyjście}
\newcommand{\interactivity}{Interakcja}
\newcommand{\grading}{Ocenianie}
\newcommand{\scoring}{Punkty}
\newcommand{\constraints}{Ograniczenia}
\renewcommand{\sampleinputname}{Przykładowe wejście}
\renewcommand{\sampleoutputname}{Przykładowe wyjście}
\newcommand{\sampleexplanation}[1]{Wyjaśnienie do przykładu #1}
\newcommand{\sampleexplanations}{Wyjaśnienie do przykładów}
\newcommand{\timelimit}{Limit czasu}
\newcommand{\memorylimit}{Limit pamięci}
\newcommand{\seconds}{s}
\newcommand{\megabytes}{MB}
\newcommand{\group}{Grupa}
\newcommand{\points}{Punkty}
\newcommand{\limitsname}{Limity}
\newcommand{\additionalconstraints}{Dodatkowe ograniczenia}
\newcommand{\testgroups}{
Zestaw testów dzieli się na kilka grup, każda jest warta pewną liczbę punktów.
Każda grupa składa się z jednego bądź większej liczby testów.
Aby otrzymać punkty za daną grupę, Twoje rozwiązanie musi przejść wszystkie testy z tej grupy.
Ostateczny wynik za zadanie jest liczony jako maksymalny wynik z pojedynczych zgłoszeń.
}



% \fi
% \newcommand{\boi}{Baltic Olympiad in Informatics}
% \newcommand{\practicesession}{Practice Session}
% \newcommand{\contestdates}{April 27 - May 1, 2018}
% \newcommand{\dayone}{Day 1}
% \newcommand{\daytwo}{Day 2}
% \newcommand{\licensingtext}{This problem is licensed under CC BY-SA 4.0.}
% \newcommand{\problem}{Problem}
% \newcommand{\inputsection}{Input}
% \newcommand{\outputsection}{Output}
% \newcommand{\interactivity}{Interactivity}
% \newcommand{\grading}{Grading}
% \newcommand{\scoring}{Scoring}
% \newcommand{\constraints}{Constraints}
% \renewcommand{\sampleinputname}{Sample Input}
% \renewcommand{\sampleoutputname}{Sample Output}
% \newcommand{\sampleexplanation}[1]{Explanation of Sample #1}
% \newcommand{\sampleexplanations}{Explanation of Samples}
% \newcommand{\timelimit}{Time Limit}
% \newcommand{\memorylimit}{Memory Limit}
% \newcommand{\seconds}{s}
% \newcommand{\megabytes}{MB}
% \newcommand{\group}{Group}
% \newcommand{\points}{Points}
% \newcommand{\limitsname}{Limits}
% \newcommand{\additionalconstraints}{Additional Constraints}
% \newcommand{\testgroups}{
% Your solution will be tested on a set of test groups, each worth a number of points.
% Each test group contains a set of test cases.
% To get the points for a test group you need to solve all test cases in the test group.
% }
\fi
\def\version{jury-1}
\problemname{Prosty Rachunek}

Dane są trzy liczby całkowite $a, b, c$ ($1 \le a, b, c \le 10^9$). Oblicz $a \cdot b / c$.
Wynik zostanie uznany za poprawny jeśli błąd bezwzględny nie będzie większy niż $10^{-6}$.

\section*{\inputsection}
W pierwszym i jedynym wierszu wejścia znajdują się trzy liczby całkowite pooddzielane pojedynczymi odstępami.

\section*{\outputsection}
Na wyjście wypisz jedną liczbę rzeczywistą, różniącą się od $a \cdot b / c$ o co najwyżej $10^{-6}$, tj. musi spełniać $|x - a \cdot b / c| \le 10^{-6}$. .

\section*{\constraints}
\testgroups

\noindent
\begin{tabular}{| l | l | l |}
\hline
\group & \points & \limitsname \\ \hline
1 & 25 & $1 \le a, b, c \le 10$ \\ \hline
2 & 25 & $1 \le a, b, c \le 10\,000$ \\ \hline
3 & 25 & $1 \le a, b \le 10^9, c = 1$ \\ \hline
4 & 25 & $1 \le a, b, c \le 10^9$ \\ \hline
\end{tabular}
