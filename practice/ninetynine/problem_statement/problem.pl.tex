\ifx\boi\undefined\ifx\problemname\undefined
\providecommand\sampleinputname{}
\providecommand\sampleoutputname{}
\documentclass[polish]{templates/boi}
\problemlanguage{.pl}
\fi
\newcommand{\boi}{Bałtycka Olimpiada Informatyczna}
\newcommand{\practicesession}{Dzień próbny}
\newcommand{\contestdates}{27 kwietnia - 1 maja 2018}
\newcommand{\dayone}{Dzień 1}
\newcommand{\daytwo}{Dzień 2}
\newcommand{\licensingtext}{To zadanie jest objęte licencją CC BY-SA 4.0.}
\newcommand{\problem}{Zadanie}
\newcommand{\inputsection}{Wejście}
\newcommand{\outputsection}{Wyjście}
\newcommand{\interactivity}{Interakcja}
\newcommand{\grading}{Ocenianie}
\newcommand{\scoring}{Punkty}
\newcommand{\constraints}{Ograniczenia}
\renewcommand{\sampleinputname}{Przykładowe wejście}
\renewcommand{\sampleoutputname}{Przykładowe wyjście}
\newcommand{\sampleexplanation}[1]{Wyjaśnienie do przykładu #1}
\newcommand{\sampleexplanations}{Wyjaśnienie do przykładów}
\newcommand{\timelimit}{Limit czasu}
\newcommand{\memorylimit}{Limit pamięci}
\newcommand{\seconds}{s}
\newcommand{\megabytes}{MB}
\newcommand{\group}{Grupa}
\newcommand{\points}{Punkty}
\newcommand{\limitsname}{Limity}
\newcommand{\additionalconstraints}{Dodatkowe ograniczenia}
\newcommand{\testgroups}{
Zestaw testów dzieli się na kilka grup, każda jest warta pewną liczbę punktów.
Każda grupa składa się z jednego bądź większej liczby testów.
Aby otrzymać punkty za daną grupę, Twoje rozwiązanie musi przejść wszystkie testy z tej grupy.
Ostateczny wynik za zadanie jest liczony jako maksymalny wynik z pojedynczych zgłoszeń.
}



% \fi
% \newcommand{\boi}{Baltic Olympiad in Informatics}
% \newcommand{\practicesession}{Practice Session}
% \newcommand{\contestdates}{April 27 - May 1, 2018}
% \newcommand{\dayone}{Day 1}
% \newcommand{\daytwo}{Day 2}
% \newcommand{\licensingtext}{This problem is licensed under CC BY-SA 4.0.}
% \newcommand{\problem}{Problem}
% \newcommand{\inputsection}{Input}
% \newcommand{\outputsection}{Output}
% \newcommand{\interactivity}{Interactivity}
% \newcommand{\grading}{Grading}
% \newcommand{\scoring}{Scoring}
% \newcommand{\constraints}{Constraints}
% \renewcommand{\sampleinputname}{Sample Input}
% \renewcommand{\sampleoutputname}{Sample Output}
% \newcommand{\sampleexplanation}[1]{Explanation of Sample #1}
% \newcommand{\sampleexplanations}{Explanation of Samples}
% \newcommand{\timelimit}{Time Limit}
% \newcommand{\memorylimit}{Memory Limit}
% \newcommand{\seconds}{s}
% \newcommand{\megabytes}{MB}
% \newcommand{\group}{Group}
% \newcommand{\points}{Points}
% \newcommand{\limitsname}{Limits}
% \newcommand{\additionalconstraints}{Additional Constraints}
% \newcommand{\testgroups}{
% Your solution will be tested on a set of test groups, each worth a number of points.
% Each test group contains a set of test cases.
% To get the points for a test group you need to solve all test cases in the test group.
% }
\fi
\def\version{jury-1}
\problemname{Dziewięćdziesiąt dziewięć}

Grasz z kolegą w grę, która nazywa sie \emph{Dziewięćdziesiąt dziewięć}.
Zaczynasz, mówiąc $1$ lub $2$. Później wykonujecie ruchy na zmianę.
Zaczyna kolega. W każdym ruchu gracz zwiększa liczbę o $1$ lub $2$.
Wygrywa gracz, który pierwszy powie $99$!

Napisz program, który gra za ciebie i wygrywa.

\section*{\interactivity}
To zadanie jest interaktywne.

Twój program zaczyna wypisując na standardowe wyjście jedną linię zawierającą $1$ lub $2$.
Program sprawdzający wczytuje tę liczbę (nazwijmy ją $x$) i wypisuje pojedynczą liczbę
zawierającą $x+1$ lub $x+2$, którą może wczytać twój program. Wtedy twój program wykonuje ruch
wypisując liczbę o $1$ lub $2$ większą i tak dalej.

Jeśli twój program wygra wypisując liczbę $99$, powinien zakończyć się z kodem wyjścia 0.
Jeśli przegra, wczytując $99$, również powinien się zakończyć; wówczas test otrzyma werdykt Wrong Answer.
Wypisanie niepoprawnej wartości (także liczby większej niż $99$) również zakończy się wynikiem Wrong Answer.
Sprawdzarka zakłada, że po wypisaniu błędnej wartości program od razu się zakończy. W przeciwnym razie
może otrzymać wynik Wrong Answer, Runtime Error lub Time Limit Exceeded
w zależności od swojego zachowania.

Pamiętaj, żeby opróżniać bufor wyjściowy \emph{po każdym ruchu}, przed wczytaniem odpowiedzi
sprawdzarki, inaczej twój program otrzyma wynik Time Limit Exceeded.
Komendy opróżniające bufor we wspieranych językach:
\begin{itemize}
  \item Java: \texttt{System.out.println()} robi to automatycznie.
  \item Python: \texttt{print()} również robi to sam.
  \item C++: \texttt{cout << endl;} opróżnia bufor, dodatkowo wypisując znak nowej linii. Jeżeli używasz printf: \texttt{fflush(stdout)}.
  \item Pascal: \texttt{Flush(Output)}.
\end{itemize}

\section*{\constraints}
\testgroups

\noindent
\begin{tabular}{| l | l | l |}
\hline
\group & \points & \constraints \\ \hline
  1      & 30     & Kolega zawsze zwiększa numer o $1$. \\ \hline
  2      & 30     & Kolega zawsze zwiększa numer o $2$ (oprócz $98$). \\ \hline
  3      & 40     & Kolega gra losowo, wybierając każdą opcję z prawdopodobieństwem 50\% (oprócz $98$). \\ \hline
\end{tabular}
