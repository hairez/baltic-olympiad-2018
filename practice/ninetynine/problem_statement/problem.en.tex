\ifx\boi\undefined\ifx\problemname\undefined
\providecommand\sampleinputname{}
\providecommand\sampleoutputname{}
\documentclass[english]{templates/boi}
\problemlanguage{.en}
\fi
\newcommand{\boi}{Baltic Olympiad in Informatics}
\newcommand{\practicesession}{Practice Session}
\newcommand{\contestdates}{April 27 - May 1, 2018}
\newcommand{\dayone}{Day 1}
\newcommand{\daytwo}{Day 2}
\newcommand{\licensingtext}{This problem is licensed under CC BY-SA 4.0.}
\newcommand{\problem}{Problem}
\newcommand{\inputsection}{Input}
\newcommand{\outputsection}{Output}
\newcommand{\interactivity}{Interactivity}
\newcommand{\grading}{Grading}
\newcommand{\scoring}{Scoring}
\newcommand{\constraints}{Constraints}
\renewcommand{\sampleinputname}{Sample Input}
\renewcommand{\sampleoutputname}{Sample Output}
\newcommand{\sampleexplanation}[1]{Explanation of Sample #1}
\newcommand{\sampleexplanations}{Explanation of Samples}
\newcommand{\timelimit}{Time Limit}
\newcommand{\memorylimit}{Memory Limit}
\newcommand{\seconds}{s}
\newcommand{\megabytes}{MB}
\newcommand{\group}{Group}
\newcommand{\points}{Points}
\newcommand{\limitsname}{Limits}
\newcommand{\additionalconstraints}{Additional Constraints}
\newcommand{\testgroups}{
Your solution will be tested on a set of test groups, each worth a number of points.
Each test group contains a set of test cases.
To get the points for a test group you need to solve all test cases in the test group.
Your final score will be the maximum score of a single submission.
}
\fi
\def\version{jury-1}
\problemname{Ninety-nine}

You and your friend are playing a game which you call \emph{Ninety-nine}.
You start, by saying either the number $1$ or $2$.
You then take turns, starting with your friend, increasing this number by either $1$ or $2$ in each step.
The first player who gets to say the number $99$ wins!

Write a program which plays this game for you and wins.

\section*{\interactivity}
This problem is interactive.

Your program should start by printing either the number $1$ or $2$ on a single line.
The grader will then read this value (call it $x$), and in return print a line with either $x+1$ or $x+2$, which can be read by your program.
Your program should then print a value which is $1$ or $2$ higher, and so on.

If your program reads $99$, it should exit normally (with status code 0); the test group will then get the verdict Wrong Answer.
If your program prints $99$, it should also exit.
If your program does not exit as it should, its verdict may be either Wrong Answer, Runtime Error or Time Limit Exceeded, depending on exact circumstances.

You \emph{must} make sure to flush standard output before reading the grader's response, or else your program
will get judged as \emph{Time Limit Exceeded}. This works as follows in the supported languages:
\begin{itemize}
  \item Java: \texttt{System.out.println()} flushes automatically.
  \item Python: \texttt{print()} flushes automatically.
  \item C++: \texttt{cout << endl} flushes, in addition to writing a newline. If using printf, \texttt{fflush(stdout)}.
  \item Pascal: \texttt{Flush(Output)}
\end{itemize}

\section*{\constraints}
\testgroups

\noindent
\begin{tabular}{| l | l | l |}
\hline
\group & \points & \constraints \\ \hline
  1      & 30     & Your friend always increases the number by 1. \\ \hline
  1      & 30     & Your friend always increases the number by 2 (unless at 98). \\ \hline
  2      & 40     & Your friend plays randomly, with each option having 50\% probability. \\ \hline
\end{tabular}
